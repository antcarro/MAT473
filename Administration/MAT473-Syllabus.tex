\documentclass[11pt]{article}
\usepackage{graphicx}
\usepackage{amsfonts,amssymb,amsmath,multirow}
\usepackage{longtable}
\usepackage[numbers,square]{natbib}
\usepackage[left=1in,top=1in,right=1in,bottom=1in,nohead]{geometry}
\usepackage{pdfsync}
\usepackage{hyperref}
\usepackage[table]{xcolor}
\usepackage{fancyhdr}
\lhead{MAT473: Graduate Rings and Modules}
\chead{}
\rhead{Winter, 2017}
\lfoot{}
\cfoot{}
\rfoot{\large\thepage}

\makeatletter
\renewcommand\subsection{%
  \@startsection{subsubsection}{3}{\z@}%
  {3.25ex\@plus 1ex \@minus .1ex}%
  {-1em}%
  {\normalfont\normalsize\bfseries}}
\makeatother


\begin{document}
\pagestyle{fancy} 



\begin{table}[h]
  \centering
  \begin{tabular}{|l|l|}
\hline
   \begin{minipage}[t]{0.5\textwidth} {\bf Instructor:} Andrew
     T. Carroll\\ {\bf Contact:}
     \href{mailto:andrew.carroll@depaul.edu}{andrew.carroll@depaul.edu}\end{minipage}
 & \begin{minipage}[t]{0.5\textwidth} {\bf
        Office:} 512 SAC \\  {\bf Office Hours:} 
      1:00pm-2:30pm Tuesday \& Thursday\\
      Or by appointment
  \end{minipage}\\ \hline
{\bf Meetings:} Tuesday 6:00pm-9:15pm & Levan 403\\ \hline
  \end{tabular}\\ 
\end{table}

\section*{Generalities} 

\subsection*{Raison d'\^etre} This course is a graduate-level
introduction to rings and modules with algebraic geometry at the end
(time-permitting). Rings are natural generalizations of some objects
you're likely quite familiar with: integers, polynomials, and
square matrices. In each context, we have (at least) two binary
operations, addition and multiplication, that play nicely
together. Generalizing these examples gives us a general framework in
which we can prove powerful results. We then move to modules, which
are very much like vector spaces but that the underlying field has
been replaced by a ring. It turns out that the structure of the
modules over a ring tells us a lot about the ring itself. 

To succeed in this course, most students will need to commit at least 8 hours per week in addition to the class meetings to work on course material. Students who are unable to commit and sustain this level of attention are unlikely to fully succeed in the course. Some well-prepared students may be able to master the material in fewer hours per week, but other students may require more.

\subsection*{Course Description}
\label{sec:course-description}

You should expect to leave this course feeling comfortable with the
basic definitions related to rings: homomorphisms, ideals, quotient
rings, rings of fractions, types of domains (Euclidean, Principal
Ideal, and Unique Factorization). You will also understand the
following important module structures: quotients, direct sums, free
modules, projective/injective modules, and exact sequences. You will
also learn about the structure theory of modules over principal ideal
domains (a class which includes the integers and a polynomial ring in
one variable over a field). Finally, time permitting, we'll discuss
the notion of categories, which furnish a unified language in which to
talk about the types of properties of interest that pop up throughout
the study of abstract algebra and beyond. 

\section*{Course Particulars}
\label{sec:course-requirements}

\subsection*{Prerequisites:} An undergraduate course in abstract algebra that included some group theory, or placement by the Director of the MS in Pure Mathematics program.

\subsection*{Attendance:} Attendance is absolutely mandatory. We have
10 class meetings total, so missing one class is missing 10\% of the
course! Attendance will be taken at each meeting. If you miss a class
without an approved excuse, you will drop one half-letter grade (about 5\%).

\subsection*{Text:} We will use the text {\bf Abstract Algebra},
3$^{rd}$ edition by D. Dummit and R. Foote. ISBN-13:
978-0471433347. I'll be upfront about this: every solution to every
problem in this book is online. Hence, I will write many of the
homework problems. The course will cover chapters 7-12 (roughly). 

\subsection*{Homework:} Each week, I will ask you to complete a
selection of 7 to 12 problems, which will be posted on D2L. You
\emph{must} submit these problems typed, preferably in \LaTeX. If you
can't download \LaTeX to your machine for some reason, you can make a
free account at \url{www.shareletex.com}. I will distribute some
practice exercises to get you up to speed on writing with \LaTeX.

I will grade typed homework by its clarity, logical consistency,
and veracity (did it actually prove what you claim?). The rubric I use
is as follows:\\

\begin{tabular}[h]{|l|l|}
  \rowcolor{cyan} Grade & Criteria \\ \hline 
2& This is correct and well-written mathematics!\\\hline
1 & \begin{minipage}[t]{0.75\textwidth} Writing is unclear and needs
  to be cleaned up, or minor errors in the mathematical reasoning. Rewrite.
  \end{minipage}\\\hline
0 & \begin{minipage}[t]{0.75\textwidth} Significant mathematical
  errors or writing that isn't comprehensible enough to grade. Rewrite.\end{minipage}\\
\hline
\end{tabular}

Part of developing a mindset for mathematics is to take constructive
criticism and use it to improve. To aid you in this development, I
will accept corrections up to 1 week after the grade for a given assignment is posted. 

Group work on homework is acceptable, but what is submitted must be
your own writing! I will give you a 0 for the whole homework
assignment if I see evidence of plagiarized work (this includes
copying solutions from online or from a classmate). If you do work
with a group, you must make a bibliographic citation for your
partners. If you looked online for some help, you must cite this as
well. If you found a solution in the book, again, citation. Give
credit where it's due.

\subsection*{Presentations:} You will be asked to present from a range
of problems each week. Each week, I'll post a collection of theorems and
unsolved examples. You will be expected to sign up for, and present,
on one of these each week. Some of these are difficult theorems, and
so you won't be solving the problem, just preparing a proof from a
book to present. Others may be homework problems that you do have to
solve. Each choice has its challenges, but you should make sure that
you're challenging yourself. Pick problems that are \emph{not} easy to
present and push yourself to understand them. You may wish to complete
many of the assigned problems so that you have a chance to present.

Your grade for these presentations will be based on the following rubric:\\

\begin{tabular}[h]{|l|l|}
  \rowcolor{cyan} Grade & Criteria\\ \hline
2 &\begin{minipage}{0.5\linewidth}Excellent, completely correct and
  clear proof or solution. \end{minipage}\\ \hline
1 & \begin{minipage}{0.5\linewidth}\ \\Communication needs improvement, or there are minor mathematical
    flaws.\end{minipage}\\\hline
0 & \begin{minipage}{0.5\linewidth}\ \\I am unconvinced that you understand what you are presenting; has
    major mathematical flaws and logic is incoherently
    presented. \end{minipage}\\ \hline
\end{tabular}

\newpage
\subsection*{Grade:} Your grade will be based on three categories:
homeworks, presentations, and exams . \\

\begin{tabular}[h]{|l|l|}
  \rowcolor{cyan} Category & Percentage of final grade\\\hline
 Typed homework & 30\%\\ \hline
Presentations & 25\%\\ \hline
Midterm & 20 \%\\ \hline
Final Exam & 25\%\\ \hline
\end{tabular}

\subsection*{Academic Integrity:} Work done for this course must adhere to the University Academic Integrity Policy, which you can review in the Student Handbook or at http://academicintegrity.depaul.edu. You are acting with academic integrity if you submit work that authentically reflects your personal understanding, however incomplete, of the material in question. Submitting work that does not authentically reflect your personal understanding, or enabling someone else to do so, is a serious violation of the Academic Integrity Policy. If I determine that you have violated the policy, I will report the incident to the Academic Integrity office, and may take additional action.

\subsection*{Accommodations:} I am committed to making reasonable
accommodations for students with disabilities or exceptional
circumstances. To ensure that you receive the most appropriate accommodations based on your needs, please contact me as early as possible,
preferably during the first week of classes. Depending on the
accommodations sought, supporting documentation from the Center for
Students with Disabilities may be required. All discussions will
remain confidential.


\end{document}

%%% Local Variables:
%%% mode: latex
%%% TeX-master: t
%%% End:


