\documentclass[12pt]{article}
\usepackage{fancyhdr}
\usepackage{amsmath,amsthm,amssymb,dsfont,enumerate,color}
\usepackage[top=1in, bottom=1in]{geometry}
\usepackage{tikz}
\usepackage{tikz-3dplot}
\usetikzlibrary{patterns}
\tdplotsetmaincoords{70}{110}
\usetikzlibrary{arrows.meta}
\fancyhead[L]{MAT473}
\fancyhead[R]{Midterm Exam}
\fancyfoot[L]{Name: \underline{\hspace{2in}}}
\fancyfoot[R]{\large \thepage}
\fancyfoot[C]{}
\newcommand{\bbA}{\mathbb{A}}
\newcommand{\bbB}{\mathbb{B}}
\newcommand{\bbC}{\mathbb{C}}
\newcommand{\bbD}{\mathbb{D}}
\newcommand{\bbE}{\mathbb{E}}
\newcommand{\bbF}{\mathbb{F}}
\newcommand{\bbG}{\mathbb{G}}
\newcommand{\bbH}{\mathbb{H}}
\newcommand{\bbI}{\mathbb{I}}
\newcommand{\bbJ}{\mathbb{J}}
\newcommand{\bbK}{\mathbb{K}}
\newcommand{\bbL}{\mathbb{L}}
\newcommand{\bbM}{\mathbb{M}}
\newcommand{\bbN}{\mathbb{N}}
\newcommand{\bbO}{\mathbb{O}}
\newcommand{\bbP}{\mathbb{P}}
\newcommand{\bbQ}{\mathbb{Q}}
\newcommand{\bbR}{\mathbb{R}}
\newcommand{\R}{\mathbb{R}}
\newcommand{\bbS}{\mathbb{S}}
\newcommand{\bbT}{\mathbb{T}}
\newcommand{\bbU}{\mathbb{U}}
\newcommand{\bbV}{\mathbb{V}}
\newcommand{\bbW}{\mathbb{W}}
\newcommand{\bbX}{\mathbb{X}}
\newcommand{\bbY}{\mathbb{Y}}
\newcommand{\bbZ}{\mathbb{Z}}
\newcommand{\bbk}{\mathbb{k}}

\newcommand{\calA}{\mathcal{A}}
\newcommand{\calB}{\mathcal{B}}
\newcommand{\calC}{\mathcal{C}}
\newcommand{\calD}{\mathcal{D}}
\newcommand{\calE}{\mathcal{E}}
\newcommand{\calF}{\mathcal{F}}
\newcommand{\calG}{\mathcal{G}}
\newcommand{\calH}{\mathcal{H}}
\newcommand{\calI}{\mathcal{I}}
\newcommand{\calJ}{\mathcal{J}}
\newcommand{\calK}{\mathcal{K}}
\newcommand{\calL}{\mathcal{L}}
\newcommand{\calM}{\mathcal{M}}
\newcommand{\calN}{\mathcal{N}}
\newcommand{\calO}{\mathcal{O}}
\newcommand{\calP}{\mathcal{P}}
\newcommand{\calQ}{\mathcal{Q}}
\newcommand{\calR}{\mathcal{R}}
\newcommand{\calS}{\mathcal{S}}
\newcommand{\calT}{\mathcal{T}}
\newcommand{\calU}{\mathcal{U}}
\newcommand{\calV}{\mathcal{V}}
\newcommand{\calW}{\mathcal{W}}
\newcommand{\calX}{\mathcal{X}}
\newcommand{\calY}{\mathcal{Y}}
\newcommand{\calZ}{\mathcal{Z}}
\newcommand{\iitem}{\vfill \item}
\newcommand{\topic}[1]{\textcolor{blue}{#1}}
\newcommand{\answerbox}{\begin{flushright}
    \begin{tikzpicture}
      \draw (0,0) rectangle (5,-1.75);
    \end{tikzpicture}\end{flushright}}
\newcommand{\solution}[1]{\textcolor{red}{#1}}
\newcommand{\points}[1]{\ [#1pts]}
%\renewcommand{\solution}[1]{}

\begin{document}
\pagestyle{fancy}

Throughout, $R$ is a ring with $1\neq 0$. When a question says
``describe'' or ``find'' or something like that, I'm still asking you
for a proof!
\begin{enumerate}
\item Prove the following equivalences for commutative rings $R$:
  \begin{enumerate}
  \item An ideal $I$ is prime if and only if $R/P$ is an integral
    domain.
  \item An ideal $I$ is maximal if and only if $R/P$ is a field.
  \end{enumerate}
\item Suppose that $I_1,I_2,\dotsc,I_n$ are two-sided ideals in a ring
  $R$. 
  \begin{enumerate}
  \item Prove that $\bigcap_{j=1}^n I_j$ is a two-sided ideal.
  \item Prove that $\prod\limits_{j=1}^n I_j$ is a two-sided
    ideal.\footnote{$I_1I_2:=\{\sum\limits_{k=1}^L r_k^{(1)}r_k^{(2)}
      \mid r_k^{(1)}\in I_1, r_k^{(2)}\in I_2$ is the product of
      ideals, and the n-fold product is defined recursively.}
  \item Determine the containment dependence between (a) and (b) (that
    is, which one must be contained in the other). 
  \end{enumerate}
\item Suppose that $\varphi: R\rightarrow S$ is a homomorphism of
  rings (assume for this problem that $\varphi(1_R)=1_S$. 
\begin{enumerate}
\item Prove that if $P$ is a prime ideal in $S$, then
  $\varphi^{-1}(P)$, its preimage in $R$, is a prime ideal in $R$. 
\item Suppose that $M$ is a maximal ideal in $S$ and that $\varphi$ is
  surjective. Prove that $\phi^{-1}(M)$ is a maximal ideal in $R$. 
\item Give an example of a homomorphism of rings $\varphi:
  R\rightarrow S$ and a maximal ideal $M\subset S$ such that
  $\varphi^{-1}(M)$ is \emph{not} a maximal ideal in $R$. 
\end{enumerate}
\item Suppose that $R$ is an integral domain and that $D\subset R$ is
  a non-empty multiplicative set not containing $0$. Prove that
  $D^{-1}R$ is also an integral domain. (Be very careful about what it
  means to equal $0$ in $D^{-1}R$.)
\item Consider the set $\bbR[[t]]=\{\sum\limits_{i\geq 0} a_i t^i\mid
  a_i \in \bbR\}$, known as the ring of formal power series in the
  variable $t$, with addition and multiplication as in the case of the
  polynomial ring. \footnote{\[\left(\sum\limits_{i\geq 0} a_i
      t^i\right) \cdot \left(\sum\limits_{i\geq 0} b_i t^i \right) =
    \sum\limits_{i\geq 0} c_i t^i\] and $c_i = \sum\limits_{j=0}^i a_jb_{i-j}$.}
  \begin{enumerate}
  \item Describe all the units in $\bbR[[t]]$. 
  \item Prove that $\bbR[[t]]$ is a local ring by finding its unique
    maximal ideal.
  \end{enumerate}

\end{enumerate}
\end{document}
2018/01/31 17:39:56
