\documentclass[12pt]{article}
\usepackage{fancyhdr}
\usepackage{amsmath,amsthm,amssymb,dsfont,enumerate,color}
\usepackage[top=1in, bottom=1in]{geometry}
\usepackage{tikz}
\usepackage{tikz-3dplot}
\usetikzlibrary{patterns}
\tdplotsetmaincoords{70}{110}
\usetikzlibrary{arrows.meta}
\fancyhead[L]{MAT473}
\fancyhead[R]{Homework 2}
\fancyfoot[L]{Name: \underline{\hspace{2in}}}
\fancyfoot[R]{\large \thepage}
\fancyfoot[C]{}
\newcommand{\bbA}{\mathbb{A}}
\newcommand{\bbB}{\mathbb{B}}
\newcommand{\bbC}{\mathbb{C}}
\newcommand{\bbD}{\mathbb{D}}
\newcommand{\bbE}{\mathbb{E}}
\newcommand{\bbF}{\mathbb{F}}
\newcommand{\bbG}{\mathbb{G}}
\newcommand{\bbH}{\mathbb{H}}
\newcommand{\bbI}{\mathbb{I}}
\newcommand{\bbJ}{\mathbb{J}}
\newcommand{\bbK}{\mathbb{K}}
\newcommand{\bbL}{\mathbb{L}}
\newcommand{\bbM}{\mathbb{M}}
\newcommand{\bbN}{\mathbb{N}}
\newcommand{\bbO}{\mathbb{O}}
\newcommand{\bbP}{\mathbb{P}}
\newcommand{\bbQ}{\mathbb{Q}}
\newcommand{\bbR}{\mathbb{R}}
\newcommand{\R}{\mathbb{R}}
\newcommand{\bbS}{\mathbb{S}}
\newcommand{\bbT}{\mathbb{T}}
\newcommand{\bbU}{\mathbb{U}}
\newcommand{\bbV}{\mathbb{V}}
\newcommand{\bbW}{\mathbb{W}}
\newcommand{\bbX}{\mathbb{X}}
\newcommand{\bbY}{\mathbb{Y}}
\newcommand{\bbZ}{\mathbb{Z}}
\newcommand{\bbk}{\mathbb{k}}

\newcommand{\calA}{\mathcal{A}}
\newcommand{\calB}{\mathcal{B}}
\newcommand{\calC}{\mathcal{C}}
\newcommand{\calD}{\mathcal{D}}
\newcommand{\calE}{\mathcal{E}}
\newcommand{\calF}{\mathcal{F}}
\newcommand{\calG}{\mathcal{G}}
\newcommand{\calH}{\mathcal{H}}
\newcommand{\calI}{\mathcal{I}}
\newcommand{\calJ}{\mathcal{J}}
\newcommand{\calK}{\mathcal{K}}
\newcommand{\calL}{\mathcal{L}}
\newcommand{\calM}{\mathcal{M}}
\newcommand{\calN}{\mathcal{N}}
\newcommand{\calO}{\mathcal{O}}
\newcommand{\calP}{\mathcal{P}}
\newcommand{\calQ}{\mathcal{Q}}
\newcommand{\calR}{\mathcal{R}}
\newcommand{\calS}{\mathcal{S}}
\newcommand{\calT}{\mathcal{T}}
\newcommand{\calU}{\mathcal{U}}
\newcommand{\calV}{\mathcal{V}}
\newcommand{\calW}{\mathcal{W}}
\newcommand{\calX}{\mathcal{X}}
\newcommand{\calY}{\mathcal{Y}}
\newcommand{\calZ}{\mathcal{Z}}
\newcommand{\iitem}{\vfill \item}
\newcommand{\topic}[1]{\textcolor{blue}{#1}}
\newcommand{\answerbox}{\begin{flushright}
    \begin{tikzpicture}
      \draw (0,0) rectangle (5,-1.75);
    \end{tikzpicture}\end{flushright}}
\newcommand{\solution}[1]{\textcolor{red}{#1}}
\newcommand{\points}[1]{\ [#1pts]}
%\renewcommand{\solution}[1]{}

\begin{document}
\pagestyle{fancy}


\begin{enumerate}
\item Recall that in order to prove that given a chain of ideals
  $J\subset I \subset R$, there is an isomorphism $(R/I)/(J/I)\cong
  R/J$, we wanted to study the function
  \begin{align*}
    \varphi: R/I &\rightarrow R/J\\
    \varphi: r+I &\mapsto r+J.
  \end{align*}
Prove that $\varphi$ is a well-defined ring homomorphism with
$\ker(\varphi)=J/I =\{j+I\mid j\in J\}$. 
\solution{
[Well-defined] Suppose that $r_1+I=r_2+I$. Equivalently, $r_1-r_2\in
I$. Then
\begin{align*}
  \varphi(r_1+I) &= r_1+J\\
  \varphi(r_2+I)&= r_2+J.
\end{align*}
But $r_1-r_2\in I\subset J$, so $r_1+J=r_2+J$. Hence,
$\varphi(r_1+I)=\varphi(r_2+I)$. 
[Homomorphism] Suppose that $r,s\in R$. Then
\begin{align*}
  \varphi((r+I)+(s+I)) &= \varphi((r+s)+I) \
&= r+s+J\\
&= (r+J)+(s+J)\\
&= \varphi(r+I)+\varphi(s+I).
\end{align*}
The multiplication is similar. 
[Kernel] We have $x+I\in \ker(\varphi)$ if and only if
$\varphi(x+I)=x+J=0+J$, which holds if and only if $x\in J$. Thus,
$\ker(\varphi)=\{x+I \mid x\in J\}$, which we denote $J/I$.
}
\item Describe all ideals in $\bbZ/n\bbZ$ for any positive integer
  $n$. 
\solution{
Recall that in any ring $R$ and for any ideal $I$, there is a
bijection between ideals $J$ containing $I$ and ideals in $R/I$. Thus,
the ideals in $\bbZ/n\bbZ$ are all associated with ideals $m\bbZ$ in
$\bbZ$ that contain $n\bbZ$. But these are precisely the ideals
$m\bbZ$ with $m\mid n$. Hence, for each $m\mid n$, there is an ideal
$m\bbZ/n\bbZ$ consisting of all multiples of $[m]$ in $\bbZ/n\bbZ$.}
\item Suppose that $R$ is an ring. Prove that $R[x]$ is an integral
  domain if and only if $R$ is an integral domain.
\solution{
First, $R$ can be viewed as a subset of $R[x]$ by inclusion as the
constant polynomials. Thus, if $R$ is not an integral domain, $R[x]$
cannot be either. On the other hand, suppose that $R$ is an integral
domain. Define the function $\deg: R[x]\rightarrow \bbZ$ by taking the
degree of a polynomial to be the largest power of $x$ that occurs in a
polynomial with non-zero coefficient. Note that the degree is
multiplicative, since the highest power of $x$ possibly occuring in $f(x)g(x)$
is $x^{\deg f+\deg g}$ and it occurs with non-zero coefficient since
its coefficient is the product of the coefficients of $x^{\deg f}$ and
$x^{\deg g}$ in $f$ and $g$ and $R$ is an integral domain. 
}
\item Suppose that $F$ is a field. Prove the following:
  \begin{enumerate}
  \item If $f(x),g(x)\in F[x]$, then there exists polynomials $q(x)$
    and $r(x)$ in $F[x]$ such that $f(x)=g(x)q(x)+r(x)$ where
    $\deg(r(x))<deg(g(x))$. 
\solution{
If $\deg f(x)<\deg g(x)$, then $q(x)=0$ and $r(x)=f(x)$
work. Otherwise, suppose that $f(x)=\sum\limits_{i=0}^N a_i x^i$ and
$g(x)=\sum\limits_{i=0}^M b_i x^i$ so that $a_N\neq 0$ and $b_M\neq
0$.  Then
\[f(x)-\left(\frac{a_N}{b_M}x^{N-M}\right) g(x)\] is a polynomial of
degree $N-1$. If $N=M$, then take this resulting polynomial to be
$r(x)$. So $f(x)=\left(\frac{a_N}{b_M}x^{N-M}\right) g(x)+r(x)$ and
$\deg r(x)<M$. If $N>M$, then by induction
$f(x)-\left(\frac{a_N}{b_M}x^{N-M}\right) g(x)=g(x)q(x)+r(x)$ for some
polynomial $r(x)$ of degree less than $M$. Hence, $f(x) =
g(x)\left[q(x)+\left(\frac{a_N}{b_M}x^{N-M}\right)x^{N-M}\right]+r(x)$
and $\deg r(x)<M$. 
}
  \item Suppose that $f(x),g(x)\in F[x]$. Prove that if
    $f(x)=g(x)q(x)+r(x)$ where $q(x),r(x)$ are the polynomials
    guaranteed from above, then
    $\gcd(f(x),g(x))=\gcd(g(x),r(x))$. \footnote{This can be iterated
      to prove that $\gcd(f(x),g(x))$ can be written as a linear
      combination of $f(x)$ and $g(x)$. }
\solution{Suppose that $d(x) \mid f(x)$ and $d(x)\mid g(x)$. But
  $r(x)=f(x)-g(x)q(x)$, so $d(x)$ divides $r(x)$. Now suppose that
  $h(x)$ divides $g(x), r(x)$. Then clearly $h(x)$ dividies
  $f(x)$. Hence, the set of divisors of $f(x)$ and $g(x)$ is the set
  as the set of divisors of $g(x),r(x)$. Thus, their maximal elements
  are equal.
}
  \end{enumerate}
\item Suppose that $R$ is a commutative local ring. I.e., a
  commutative ring with
  $1\neq 0$ which
  has a unique maximal ideal, which we'll call $\mathfrak{m}$. Prove
  that $R^* = R\setminus \mathfrak{m}$. 
\solution{Suppose that $x\in R\setminus \mathfrak{m}$, and take $(x)$
  the ideal generated by $x$. Since $x\notin \mathfrak{m}$,
  $(x)\not\subset \mathfrak{m}$. But $\mathfrak{m}$ contains all
  proper ideals, so $(x)=R$. Thus, there is an element $y\in R$ such
  that $yx=1$. Hence $x$ is invertible. Now suppose that $x$ is
  invertible. Then $x\notin \mathfrak{m}$ since otherwise
  $1=x^{-1}x\in \mathfrak{m}$, implying $\mathfrak{m}=R$, but this is
  not the case.}
\item Suppose that $\mathfrak{p}$ is an ideal in a commutative ring
  with identity $1\neq 0$. Prove that $\mathfrak{p}$ is prime if and only if
    $R/\mathfrak{p}$ is an integral domain. 
\solution{Suppose that $p$ is a prime ideal, and let $x, y\in
  R$ such that $(x+p)(y+p)=0+p$ in $R/p$. Then $xy\in p$. By
  definition, then, either $x\in p$ or $y\in p$. Hence, $x+p=0$ or
  $y+p=0$. 
Now suppose that $R/p$ is an integral domain, and $x,y\in R$ such that
$xy\in p$. Then $(x+p)(y+p)=0+p$, and by assumption, one of the two
must be zero. Hence, for example, $x+p=0$ implies $x\in p$.}
\item Suppose that $n,m\in \bbZ$, and consider the function 
  \begin{align*}
    \varphi: \bbZ & \rightarrow \bbZ/n\bbZ \times \bbZ/m\bbZ\\
    \varphi: x & \mapsto ([x]_n, [x]_m)
  \end{align*}
where $[x]_n$ indicates the equivalence class of $x$ modulo $n$.
\begin{enumerate}
\item Prove that $\varphi$ is a ring homomorphism.
\solution{Projection maps are homomorphisms and products of
  homomorphisms are homomorphisms.}
\item Compute the kernel of $\varphi$ and determine the cardinality of
  its image. 
\newcommand{\lcm}{\operatorname{lcm}}
\solution{Suppose $x\in \ker \varphi$. Then $[x]_n=0$ and
  $[x]_m=0$. Hence, $x$ is a multiple of both $n$ and $m$. Thus, $\ker
  \varphi=(\lcm(n,m))\bbZ$. Note that the cardinality of the image is,
therefore, equal to the cardinality of $\bbZ/\lcm(n,m)\bbZ$, which is
precisely $\lcm(n,m)$.}
\item What does this tell you in the case that $\gcd(n,m)=1$. 
\solution{In this case, the cardinality of the image is $nm$, which is
precisely the cardinality of the codomain. Hence, $\varphi$ is onto,
so $\bbZ/nm\bbZ \cong \bbZ/n\bbZ \times \bbZ/m\bbZ$ as rings.}
\end{enumerate}


\end{enumerate}
\end{document}
2018/01/11 21:36:40
