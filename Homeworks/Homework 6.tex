\documentclass[12pt]{article}
\usepackage{fancyhdr}
\usepackage{amsmath,amsthm,amssymb,dsfont,enumerate,color}
\usepackage[top=1in, bottom=1in]{geometry}
\usepackage{tikz}
\usepackage{tikz-3dplot}
\usetikzlibrary{patterns}
\tdplotsetmaincoords{70}{110}
\usetikzlibrary{arrows.meta}
\fancyhead[L]{MAT473}
\fancyhead[R]{Homework 6}
\fancyfoot[L]{Name: \underline{\hspace{2in}}}
\fancyfoot[R]{\large \thepage}
\fancyfoot[C]{}
\newcommand{\bbA}{\mathbb{A}}
\newcommand{\bbB}{\mathbb{B}}
\newcommand{\bbC}{\mathbb{C}}
\newcommand{\bbD}{\mathbb{D}}
\newcommand{\bbE}{\mathbb{E}}
\newcommand{\bbF}{\mathbb{F}}
\newcommand{\bbG}{\mathbb{G}}
\newcommand{\bbH}{\mathbb{H}}
\newcommand{\bbI}{\mathbb{I}}
\newcommand{\bbJ}{\mathbb{J}}
\newcommand{\bbK}{\mathbb{K}}
\newcommand{\bbL}{\mathbb{L}}
\newcommand{\bbM}{\mathbb{M}}
\newcommand{\bbN}{\mathbb{N}}
\newcommand{\bbO}{\mathbb{O}}
\newcommand{\bbP}{\mathbb{P}}
\newcommand{\bbQ}{\mathbb{Q}}
\newcommand{\bbR}{\mathbb{R}}
\newcommand{\R}{\mathbb{R}}
\newcommand{\bbS}{\mathbb{S}}
\newcommand{\bbT}{\mathbb{T}}
\newcommand{\bbU}{\mathbb{U}}
\newcommand{\bbV}{\mathbb{V}}
\newcommand{\bbW}{\mathbb{W}}
\newcommand{\bbX}{\mathbb{X}}
\newcommand{\bbY}{\mathbb{Y}}
\newcommand{\bbZ}{\mathbb{Z}}
\newcommand{\bbk}{\mathbb{k}}
\newcommand{\Hom}{\operatorname{Hom}}
\newcommand{\im}{\operatorname{im}}

\newcommand{\calA}{\mathcal{A}}
\newcommand{\calB}{\mathcal{B}}
\newcommand{\calC}{\mathcal{C}}
\newcommand{\calD}{\mathcal{D}}
\newcommand{\calE}{\mathcal{E}}
\newcommand{\calF}{\mathcal{F}}
\newcommand{\calG}{\mathcal{G}}
\newcommand{\calH}{\mathcal{H}}
\newcommand{\calI}{\mathcal{I}}
\newcommand{\calJ}{\mathcal{J}}
\newcommand{\calK}{\mathcal{K}}
\newcommand{\calL}{\mathcal{L}}
\newcommand{\calM}{\mathcal{M}}
\newcommand{\calN}{\mathcal{N}}
\newcommand{\calO}{\mathcal{O}}
\newcommand{\calP}{\mathcal{P}}
\newcommand{\calQ}{\mathcal{Q}}
\newcommand{\calR}{\mathcal{R}}
\newcommand{\calS}{\mathcal{S}}
\newcommand{\calT}{\mathcal{T}}
\newcommand{\calU}{\mathcal{U}}
\newcommand{\calV}{\mathcal{V}}
\newcommand{\calW}{\mathcal{W}}
\newcommand{\calX}{\mathcal{X}}
\newcommand{\calY}{\mathcal{Y}}
\newcommand{\calZ}{\mathcal{Z}}
\newcommand{\iitem}{\vfill \item}
\newcommand{\topic}[1]{\textcolor{blue}{#1}}
\newcommand{\answerbox}{\begin{flushright}
    \begin{tikzpicture}
      \draw (0,0) rectangle (5,-1.75);
    \end{tikzpicture}\end{flushright}}
\newcommand{\solution}[1]{\textcolor{red}{#1}}
\newcommand{\points}[1]{\ [#1pts]}
%\renewcommand{\solution}[1]{}

\begin{document}
\pagestyle{fancy}


\begin{enumerate}
\item The following question gives a criterion for determining if $M$
  is a direct sum. Let $M$ be an $R$-module and $L$ a
  submodule of $M$. Let $\pi: M\rightarrow M/L$ be the projection
  homomorphisms ($\pi: m\mapsto m+L$). Suppose that there exists a
  homomorphism $p: M/L \rightarrow M$ such that $\pi \circ p=\operatorname{id}_{M/L}$ (such a function is called a
  \emph{section} of $\pi$). Prove that $M \cong L\oplus
  \operatorname{image}(p)$, and conclude that $M\cong L\oplus M/L$. 
\solution{
Suppose that $\pi \circ p = \operatorname{id}_{M/L}$. Note that
$\im(p)$ and $L$ are both submodules of $M$. To prove that $M\cong
\im(p)\oplus L$, we need only show that (i) $\im(p)\cap L =\{0\}$ and
(ii) $\im+L = M$.
\begin{itemize}
\item[i.] Suppose that $l\in L$ and $l\in \im(p)$. Hence, there exists
  an element $n+L\in M/L$ with $p(n+L) = l$.
  \begin{align*}
    \pi(l) &= \pi(p(n+L))\\
0 & = \operatorname{id}_{M/L}(n+L)\\
0 &= n+L
  \end{align*}
[here, the first line is simply expressing $l$ as $p(n+L)$, the second
is recognizing that $\pi(L)=0$ and that $\pi\circ p$ is the identity.]
Hence, $n\in L$, and so $p(n+L)=0$. Therefore, $l=0$. 
\item[ii.] Let $m\in M$. Denote by $m'$ the composition $p\circ \pi$
  applied to $m$. I.e., $m'=p(\pi(m))$. Note that
  \begin{align*}
    \pi(m') &= \pi (p(\pi(m)))\\
\pi(m') &= \operatorname{id}_{M/L} \pi(m)\\
\pi(m')&= \pi(m).
m' + L &= m+L
  \end{align*}
In particular, $m-m'\in L$. Therefore, $m=m'+l$ for some $l\in
L$. Thus, $L+\im(p)=M$. 
\end{itemize}
By these two statement, $M\cong \im(p)\oplus L$. Finally, since $\pi \circ p$
is injective (it's equal to the identity), we must have $p$ is
injective, so $\im(p)$ is isomorphic to $M/L$. Thus, $M\cong M/L
\oplus L$.}

\item Recall that if $M$ and $N$ are $R$-modules, then \[\Hom_R(M,N) =
  \{\varphi: M\rightarrow N \mid \varphi \textrm{ is a homomorphisms
    of $R$-modules}\}.\] Prove the following:
\begin{itemize}
\item If $R$ is commutative, $\Hom_R(R, M) \cong M$ (in particular, you have to show that
  $\Hom_R(R,M)$ is an $R$ module in this case). 
\solution{
Let $G: \Hom_R(R,M)\rightarrow M$ be the function with $G(\varphi) =
\varphi(1)$.
\begin{enumerate}
\item $G(\varphi+r\varphi') = \varphi(1)+r\varphi'(1) =
  G(\varphi)+rG(\varphi')$, so $G$ is indeed a homomorphism.
\item We showed in a previous homework that $\Hom_R(R,M)$ is an
  $R$-module (when $R$ is commutative). 
\item Now assume that $G(\varphi)=0$. Then $\varphi(1)=0$, so
  $\varphi(r) = r\varphi(1) = 0$ for all $r\in R$, so $\varphi=0$.
\item Finally, let $m\in M$. Define the function $\varphi_m(r) =
  rm$. We've shown previously that this function is a homomorphism,
  and clearly $G(\varphi_m) = m$, so $G$ is onto. 
\item Thus, $G$ is an isomorphism. 
\end{enumerate}
}
\item $\Hom_R(N\oplus L, M) \cong \Hom_R(N,M)\oplus \Hom_R(L,M)$
\solution{
  Define $\Psi: \Hom_R(N,M) \oplus \Hom_R(L,M) \rightarrow
  \Hom_R(N\oplus L, M)$ via $\Psi(f,g) (n,l) = f(n)+g(l)$. We
  claim this is an isomorphism (of abelian groups, since these aren't
  necessarily modules).
  \begin{enumerate}
  \item $\Psi( (f,g)+(f',g'))(n,l)= \Psi((f+f',g+g'))(n,l)$ and
    $(f+f')(n)+(g+g')(l) = f(n)+g(l) + f'(n)+g'(l) =
    \Psi(f,g)+\Psi(f',g')$.
  \item $\ker\Psi=\{(f,g) \mid f(n)+g(l) = 0 \forall n\in N, l\in L\},$
    but since $l=0\in L$, $f(n)+0=0$ for all $n\in N$, so $f=0$, and
    similarly, since $n=0\in N$, $0+g(l)=0$ for all $l\in L$, so $f=0$
    and $g=0$. 
  \item If $H: N\oplus L \rightarrow M$, then we have maps $H_N:
    N\rightarrow M$ and $H_L: L\rightarrow M$ constituted by composing
    the inclusion homomorphisms with $H$. Then $\Psi((H_N, H_L))(n,l)
    = H(n,l).$
  \end{enumerate}
}
\item If $F$ is free and $R$ is commutative, then $\Hom_R(F,R)\cong
  F$. [Note, in the non-commutative case, this is not true! Ask about
  it in class.]
\solution{Let $F$ be free with basis $A=\{f_i \mid i\in I\}$. A cute
  approach is to use the universal property of the free modules (D\&F p354): For
  any element $x$ in $F$, write $x=\sum x_i f_i$ which is, by
  definition, a finite sum. This gives rise to a set function
  $\varphi: A\rightarrow F$ defined by $\varphi(f_i) = x_i$. By
  the universal property, there is a homomorphism $\Phi: F\rightarrow
  R$ with the property that $\Phi(\sum b_i f_i) = \sum b_i
  \varphi(f_i) = \sum b_i x_i$. So this is the function we'll
  use: For any $x\in F$, let $\Phi(x): F\rightarrow R$ be the function
  defined by $\Phi(x)(\sum_i b_i f_i) = \sum_i b_i x_i$ where $x=\sum
  x_i f_i$.
  \begin{enumerate}
  \item This function is one-to-one: Assume $\Phi(x)=\Phi(y)$. Then
    $\Phi(x)(f_i) = x_i \cdot 1$ and $\Phi(y)(f_i) = y_i\cdot 1$. By
    assumption, then $x_i = y_i$ for all $i$, so $x=y$. 
  \item To prove that the function is onto, suppose that $\phi: F
    \rightarrow R$. Let $x_i = \phi(f_i)$ for each $i$, and $x=\sum_i
    x_i f_i$. Since $F$ is of finite rank, the index set is
    finite. Then $\Phi(x)(y) = \sum_i x_i y_i = \sum_i y_i \phi(f_i) =
    \sum_i \phi(y_i f_i)= \phi(\sum_i y_i f_i) = \phi(y)$. Hence,
    $\Phi(x)=\phi$. 
  \end{enumerate}
}
\end{itemize}
\item Suppose that $M$ and
  $N$ are $R$-modules which are free of finite rank, and let $\varphi:
  M\rightarrow N$ be an $R$-module homomorphism. Let
  $\{m_1,\dotsc, m_s\}$ be a basis for $M$ and $\{n_1,\dotsc, n_t\}$ a
  basis for $N$. Prove that $\phi$ can be encoded as multiplication by
  a $t\times s$ matrix with entries in $R$. \footnote{This is
    intentionally vague. Think about it for a bit, try to think about
    linear algebra, and ask me if you have questions after mulling it
    over. }
\solution{
Let $A\in \operatorname{Mat}_{t,s}(R)$ be the table of elements defined in the following way: for
each $j=1,\dotsc, s$, express $\varphi(m_j)$ as a linear combination
of the basis element of $N$, i.e., $\varphi(m_j) = \sum_i \alpha_{ij}
n_i$. If $x=\sum_j x_j m_j$, then
\begin{align*}
  \varphi(x) &= \varphi(\sum_j x_j m_j)\\
 &= \sum_j x_j \varphi(m_j)\\
&= \sum_j x_j \sum_i \alpha_{ij} n_i\\
&= \sum_i \left( \sum_j \alpha_{ij} x_j \right) n_i\\
&= \sum_i \left( A \cdot \begin{bmatrix} x_1\\ x_2 \\ \vdots \\
    x_s\end{bmatrix} \right)_i n_i
\end{align*}
So the column vector of $\varphi(x)$ is given by $A[x]$ where $[x]$ is
the column vector of $x$ in the basis $\{m_1,\dotsc, m_s\}$. 
}
\item Dummit \& Foote \S 10.3 \#12 \solution{As everyone pointed out,
    this was a previous exercise. }

\item Dummit \& Foote \S 10.3 \#27 (This one is really wild: we showed
  in linear algebra that the dimension of a vector space is unique. In
  particular, any two bases have the same cardinality. It would
  certainly be weird if we could find a 2D vector space that was
  isomorphic to a 3D vector space. But that exact thing can happen
  with non-commutative rings.)
\solution{Let $\varphi_i$ and $\psi_i$ be as defined in Dummit \&
  Foote. Note that
  \begin{align*}
    \psi_1 \phi_i(a_1,a_2,\dotsc) + \psi_2\phi_2(a_1,a_2,\dotsc ) &=
                                                                    (a_1,
                                                                    0,
                                                                    a_3,
                                                                    0,
                                                                    \dotsc
                                                                    )
                                                                    +
                                                                    (0,
                                                                    a_2,
                                                                    0,
                                                                    a_4,
                                                                    \dotsc
                                                                    )\\
 &= (a_1, a_2, a_3, \dotsc) \\
&= \operatorname{id}(a_1,\dotsc)        
  \end{align*}
so $\psi_1 \phi_1 + \psi_2 \phi_2 = id$. In particular, if
$g\in M$, then $g=(g\psi_1)\phi_1 + (g\psi_2)\phi_2$. Hence,
$\{\phi_1, \phi_2\}$ span $M$. Furthemore, suppose that $g_1 \phi_1 +
g_2 \phi_2=0$. Then 
\begin{align*}
(g_1 \phi_1+g_2\phi_2)(\psi_i) &=0\\
 g_1 \phi_1 \psi_i + g_2 \phi_2
                                 \psi_i &= 0\\
 g_i & = 0.
\end{align*}
where we used the relation that $\phi_i \psi_i=1$ and
$\phi_i\psi_{\neq i} = 0$. }
\end{enumerate}
\end{document}
2018/02/21 19:12:34
