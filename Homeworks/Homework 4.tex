\documentclass[12pt]{article}
\usepackage{fancyhdr}
\usepackage{amsmath,amsthm,amssymb,dsfont,enumerate,color}
\usepackage[top=1in, bottom=1in]{geometry}
\usepackage{tikz}
\usepackage{tikz-3dplot}
\usetikzlibrary{patterns}
\tdplotsetmaincoords{70}{110}
\usetikzlibrary{arrows.meta}
\fancyhead[L]{MAT473}
\fancyhead[R]{Homework 4}
\fancyfoot[R]{\large \thepage}
\fancyfoot[C]{}
\newcommand{\bbA}{\mathbb{A}}
\newcommand{\bbB}{\mathbb{B}}
\newcommand{\bbC}{\mathbb{C}}
\newcommand{\bbD}{\mathbb{D}}
\newcommand{\bbE}{\mathbb{E}}
\newcommand{\bbF}{\mathbb{F}}
\newcommand{\bbG}{\mathbb{G}}
\newcommand{\bbH}{\mathbb{H}}
\newcommand{\bbI}{\mathbb{I}}
\newcommand{\bbJ}{\mathbb{J}}
\newcommand{\bbK}{\mathbb{K}}
\newcommand{\bbL}{\mathbb{L}}
\newcommand{\bbM}{\mathbb{M}}
\newcommand{\bbN}{\mathbb{N}}
\newcommand{\bbO}{\mathbb{O}}
\newcommand{\bbP}{\mathbb{P}}
\newcommand{\bbQ}{\mathbb{Q}}
\newcommand{\bbR}{\mathbb{R}}
\newcommand{\R}{\mathbb{R}}
\newcommand{\bbS}{\mathbb{S}}
\newcommand{\bbT}{\mathbb{T}}
\newcommand{\bbU}{\mathbb{U}}
\newcommand{\bbV}{\mathbb{V}}
\newcommand{\bbW}{\mathbb{W}}
\newcommand{\bbX}{\mathbb{X}}
\newcommand{\bbY}{\mathbb{Y}}
\newcommand{\bbZ}{\mathbb{Z}}
\newcommand{\bbk}{\mathbb{k}}
\newcommand{\Tor}{\operatorname{Tor}}
\newcommand{\Hom}{\operatorname{Hom}}
\newcommand{\im}{\operatorname{im}}

\newcommand{\calA}{\mathcal{A}}
\newcommand{\calB}{\mathcal{B}}
\newcommand{\calC}{\mathcal{C}}
\newcommand{\calD}{\mathcal{D}}
\newcommand{\calE}{\mathcal{E}}
\newcommand{\calF}{\mathcal{F}}
\newcommand{\calG}{\mathcal{G}}
\newcommand{\calH}{\mathcal{H}}
\newcommand{\calI}{\mathcal{I}}
\newcommand{\calJ}{\mathcal{J}}
\newcommand{\calK}{\mathcal{K}}
\newcommand{\calL}{\mathcal{L}}
\newcommand{\calM}{\mathcal{M}}
\newcommand{\calN}{\mathcal{N}}
\newcommand{\calO}{\mathcal{O}}
\newcommand{\calP}{\mathcal{P}}
\newcommand{\calQ}{\mathcal{Q}}
\newcommand{\calR}{\mathcal{R}}
\newcommand{\calS}{\mathcal{S}}
\newcommand{\calT}{\mathcal{T}}
\newcommand{\calU}{\mathcal{U}}
\newcommand{\calV}{\mathcal{V}}
\newcommand{\calW}{\mathcal{W}}
\newcommand{\calX}{\mathcal{X}}
\newcommand{\calY}{\mathcal{Y}}
\newcommand{\calZ}{\mathcal{Z}}
\newcommand{\iitem}{\vfill \item}
\newcommand{\topic}[1]{\textcolor{blue}{#1}}
\newcommand{\answerbox}{\begin{flushright}
    \begin{tikzpicture}
      \draw (0,0) rectangle (5,-1.75);
    \end{tikzpicture}\end{flushright}}
\newcommand{\solution}[1]{\textcolor{red}{#1}}
\newcommand{\points}[1]{\ [#1pts]}
%\renewcommand{\solution}[1]{}

\begin{document}
\pagestyle{fancy}

Throughout, unless otherwise stated, $R$ is a ring with identity and
$M$ is a (left) $R$-module.

\begin{enumerate}
\item Suppose that $I$ is an ideal in $R$, and define by 
  \[IM=\left\{\sum\limits_{finite} a_i m_i \mid a_i\in I,\ m_i\in
      M\right\}.\] Prove that $IM$ is a submodule of $M$.
\solution{
Suppose that $A=\sum_{k\in K} a_k m_k$ and $B=\sum_J b_j m_j$ are elements of
$IM$ where $a_k, b_j\in I$ and $K$ and $J$ are finite index sets. Then
\begin{align*}
  A+rB &= \sum_{k\in K} a_k m_k + r\sum_{j\in J} b_j m_j\\
&= \sum_{l \in K\cup J} (a_l+rb_l)m_l
\end{align*}
which is in $IM$ since $al+rbl\in I$ and $K\cup J$ is finite. 
} 
\item Let $M$ and $M'$ be modules over $R$. Denote by 
  \[\Hom_{R}(M,M')=\{f: M\rightarrow M' \mid
    f(rm)=r\cdot f(m) \forall r\in R,\ m\in M\}.\] Prove that if
  $R$ is a commutative ring, then $\Hom_R(M,M')$ is an $R$-module. 
\solution{
First, we have seen in group theory that $\Hom_R(M,M')$ is an abelian
group, so we only need to demonstrate the action of $R$ on this
set. Let $\varphi\in \Hom_R(M,M')$ and $r\in R$. Define $[r\varphi]:
M\rightarrow M'$ by $[r\varphi](m) = r (\varphi(m))$. Then
\begin{enumerate}
\item $[r\varphi](m+n) =
  r(\varphi(m+n))=r\varphi(m)+r\varphi(n)=[r\varphi](m)+[r\varphi](n)$
\item $[r\varphi](sm) = r\varphi(sm)=rs\varphi(m)=sr\varphi(m)=s[r\varphi](m)$
\end{enumerate}
Hence, $[r\varphi]$ is a module homomorphism (you can check the other conditions). }
\item Suppose that $R=\bbR[t]$, and consider the space
  $M=\bbR^2$. Recall that we can make $M$ into an $R$-module by
  providing a linear transformation $T$ on $M$. To this end, let $T$
  be the transformation given by the matrix 
  \[\begin{bmatrix} 2 & 6 \\ 0 & -1\end{bmatrix}.\] Find all
  $R$-submodules of this module $M$. 
\solution{
Suppose that $N\subset M$ is a submodule. This means that $rn\in N$
for all $n\in N$ and $r\in R$, and that $n+n'\in N$ whenever $n,n'\in
N$. In particular, we consider the constant polynomials $c\in R$. This
menas that $cn\in N$ for all $n\in N$. The rest of the module axioms
show that $N$ must be a vector subspace of $M$. Since $M$ is
two-dimensional over $\bbR$, its only subspaces are $0, M$ and
$\operatorname{span}(v)$ where $v\in M$. 
Next, we consider the polynomial $t$. We require that $tv\in
\operatorname{span}(v)$, so $tv=\lambda v$ for some $\lambda \in
\bbR$. This holds if and only if $v$ is an Eigenvector of $T$. The
Eigenvectors of this matrix are $(1,0)$ and $(-2,1)$. Hence, the
submodules are $\operatorname{span}((1,0))$ and
$\operatorname{span}((-2,1))$. $t$ acts on the first via
multiplication by 2, and $t$ acts on the second via multiplication by
$-1$. 
}
\item Suppose that $M$ and $M'$ are $R$-modules and $f: M\rightarrow
  M'$ is a module homomorphism.
  \begin{enumerate}
  \item Prove that $\ker f$ is a submodule of $M$.
\solution{We know that the kernel of a group homomorphism, so $\ker f$
  is an abelian group. Furthermore, if $r\in R$ and $x\in \ker f$,
  then $f(rx) = rf(x) = r\cdot 0=0$. So $\ker f$ is a submodule of $M$.}
  \item Prove that $\im f$ is a submodule of $M'$.
\solution{Similarly, from group theory, the image of a homomorphism is
  a subgroup of the codomain. So suppose that $x\in \im f$, so
  $x=f(y)$ for some $y\in M$. Then $rx=rf(y)=f(ry)$, so $rx\in \im
  f$. }
  \item Prove that $M/\ker f\cong \im f$ (you'll have to construct the
    isomorphism, of course, but you can assume that $M/N$ is a module
    whenever $N$ is a submodule of $M$). 
\solution{
Consider the function $F: M/\ker f \rightarrow M'$ defined via
$F(m+\ker f) = f(m)$. We already know that $F$ is a well-defined group
isomorphism from the first isomorphism theorem from group theory. We
need only show, then, that $F$ is also a module homomorphism. To this
end, $F(r(m+ker f)) = F(rm+\ker f) = f(rm) = rf(m) = rF(m+\ker f)$, so
indeed, $F$ is an $R$-module isomorphism.
}
  \end{enumerate}
\item $M$ is said to be \emph{irreducible} if its only submodules are
  $0$ and $M$ itself.
  \begin{enumerate}
  \item Prove that a non-trivial module $M$ is irreducible if and only
    if $M$ can be generated by any of its non-zero
    elements.\footnote{We say $x$ generates $M$ if $M=\{rx\mid r\in
      R\}$.}
\solution{Suppose that $M$ is irreducible and  $x\neq 0$ in $M$. Then
  $Rx$ is a submodule. Since $x\neq 0$, then $x\in Rx$ so $Rx\neq
  0$. Hence, $Rx=M$. On the other hand, suppose that $M$ can be
  generated by any of its non-zero elements, and let $N\subset M$ be a
  non-zero submodule. Then there is an element $x\in N$, so by
  assumption, $M=Rx$. But $Rx\subset N$, so $N=M$. Hence, $M$ is irreducible.}
  \item Prove that if $M, N$ are irreducible modules, and
    $f:M\rightarrow N$ is a non-zero homomorphism, then $f$ is an
    isomorphism.
\solution{If $M$ and $N$ are irreducible and $f: M\rightarrow N$ is a
  module homomorphism. Then $\ker f$ and $\im f$ are submodules of $M$
  and $N$ as shown above. Since $f$ is non-zero, $\ker f \neq M$, so,
  since $M$ is irreducible, $\ker f=0$ (i.e., $f$ is
  injective). Similarly, $\im f\neq 0$, so $\im f = N$.}
  \item Describe the irreducible modules over $\bbZ$ that have finite
    cardinality. 
\solution{
Suppose that $M$ is an irreducible module over $\bbZ$ of finite
cardinality. We have seen that $\bbZ$ modules are precisely the same
as abelian groups, so $M$ is a finite abelian group. By the
classification theorem of finite abelian groups, we have $M\cong
\bigoplus\limits_{i=1}^u \bbZ/k_i\bbZ$ where $k_i \mid k_{i+1}$. In
particular, if $M$ is irreducible, it can only have one direct
summand, so $M\cong \bbZ/k \bbZ$ for some integer $k$. But if $n\mid
k$, then $n\bbZ/k\bbZ$ is a non-trivial submodule, so $\bbZ/k\bbZ$ is
simple if and only if $k=p$ is a prime. }
  \end{enumerate}

\item Consider the quiver discussed in class: $1\xrightarrow{a}
  2$. Recall that $\bbR Q$ has a basis $\{x_{e_1}, x_a,
  x_{e_2}\}$.
  \begin{enumerate}
  \item This ring has an identity element. Find it.
\solution{
$(x_{e_1}+x_{e_2})()(\gamma_1 x_{e_1}+\gamma_a x_a + \gamma_2 x_{e_2} )
=(\gamma_1 x_{e_1}+\gamma_a x_a + \gamma_2 x_{e_2})$, so
$x_{e_1}+x_{e_2}$ is the identity.  }
  \item It turns out that $\bbR Q$ is isomorphic to $T_2(\bbR)$ (the
    ring of upper-triangular matrices with entries in $\bbR$). Find
    the isomorphism and prove your result. 
\solution{
Consider the function $F: \bbR Q \rightarrow T_2(\bbR)$ defined by 
\[F(\gamma_1 x_{e_1}+\gamma_a x_a + \gamma_2 x_{e_2} )
  = \begin{bmatrix} \gamma_2 & \gamma_a \\ 0 &
    \gamma_1\end{bmatrix}.\]
The linearity is easy to verify, so you should do it. Additionally,
\begin{align*}
  F((\gamma_1 x_{e_1}+\gamma_a x_a + \gamma_2 x_{e_2})(\alpha_1
  x_{e_1}+\alpha_a x_a + \alpha_2 x_{e_2} )) &= F(\gamma_1 \alpha_1
                                               X_{e_1} \\ & \ +
                                                            (\gamma_a
                                                            \alpha_1+\alpha_a
                                                            \gamma_2)
                                                            x_a \\ & \
                                                                     +
                                                                     \gamma_2
                                                                     \alpha_2
                                                                     X_{e_2}\\
&= \begin{bmatrix} \gamma_2 \alpha_2 & \gamma_a \alpha_1 + \alpha_a
  \gamma_2 \\ 0 & \gamma_1 \alpha_1 \end{bmatrix}\\ &= \begin{bmatrix}
  \gamma_2 & \gamma_a \\ 0 & \gamma_1\end{bmatrix}
                             \cdot \begin{bmatrix} \alpha_2 & \alpha_a
                               \\ 0 & \alpha_1\end{bmatrix} .
\end{align*}
}
  \end{enumerate}

\item Consider the groups $G=\bbZ/4\bbZ$ (the cyclic group of order 4) and $H=\bbZ/2\bbZ \times
  \bbZ/2\bbZ$ (the Klein 4-group).
  \begin{enumerate}
  \item Construct the group rings $\bbR G$ and $\bbR H$ for each. That
    is, describe a basis for each and the multiplicative structure on
    this basis.
\solution{$RG$ has basis $\{x_0, x_1,x_2, x_3\}$ in which $x_0=1$,
  $x_1^2=x_2$, and $x_1^3=x_3$, and $x_1^4=1$. So
  $RG=\bbR[x]/(x^4-1)$. 
$\bbR H$ has basis $\{x_{00}, x_{10}, x_{01}, x_{11}\}$. But notice
that $x_{10}^2=x_{01}^2=1$, and $x_{11}=x_{10}x_{01}=x_{01}x_{10}$. If
we rename $x=x_{10}$ and $y=y_{01}$, then this ring is really
$\bbR[x,y]/(x^2-1, y^2-1)$. }
  \item Prove that $\bbR G$ is not isomorphic to $\bbR H$. 
\solution{
Here's one way you can go about it: We've phrased each of the rings as
quotients of polynomial rings by ideals. We could ask what the maximal
ideals are, since those are relatively easy to understand in the
context of polynomial rings. 
One theorem we have is that there is a bijection between ideals in
$R/I$ and ideals in $R$ containing $J$. So which ideals $J$ contain
$(x^4-1)$. Well we note that $x^4-1=(x-1)(x+1)(x^2+1)$. If $J$ is an
ideal containing $(x^4-1)$, then it must be generated by
$(x-1)^n(x+1)^m(x^2+1)^l$ where each of $n,m,l$ could be either 0 or
1, but no larger. This gives us a total of 8 ideals in $\bbR G$ (one
of which is $\bbR G$ itself by taking $n=m=l=0$). 
On the other hand, $(x^2-1)=(x+1)(x-1)$ and $y^2-1=(y+1)(y-1)$. Thus,
the ideals that contain $(x^2-1,y^2-1)$ are the ideals $I$ of the form $(
(x-1)^n(x+1)^m, (y-1)^l(y+1)^p)$ where each of $n$ and $m$ can be $0$
or $1$, but note that if $n=m=0$ or $l=p=0$, then $I=\bbR H$, so we're
counting that one twice. This yields a total of $2^4-1=15$
ideals. Hence, $\bbR H$ and $\bbR G$ are non-isomorphic. }
  \end{enumerate}



\end{enumerate}
\end{document}
2018/02/07 22:10:30
