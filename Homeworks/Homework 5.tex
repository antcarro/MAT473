\documentclass[12pt]{article}
\usepackage{fancyhdr}
\usepackage{amsmath,amsthm,amssymb,dsfont,enumerate,color}
\usepackage[top=1in, bottom=1in]{geometry}
\usepackage{tikz}
\usepackage{tikz-3dplot}
\usetikzlibrary{patterns}
\tdplotsetmaincoords{70}{110}
\usetikzlibrary{arrows.meta}
\fancyhead[L]{MAT473}
\fancyhead[R]{Homework 5}
\fancyfoot[L]{Name: \underline{\hspace{2in}}}
\fancyfoot[R]{\large \thepage}
\fancyfoot[C]{}
\newcommand{\bbA}{\mathbb{A}}
\newcommand{\bbB}{\mathbb{B}}
\newcommand{\bbC}{\mathbb{C}}
\newcommand{\bbD}{\mathbb{D}}
\newcommand{\bbE}{\mathbb{E}}
\newcommand{\bbF}{\mathbb{F}}
\newcommand{\bbG}{\mathbb{G}}
\newcommand{\bbH}{\mathbb{H}}
\newcommand{\bbI}{\mathbb{I}}
\newcommand{\bbJ}{\mathbb{J}}
\newcommand{\bbK}{\mathbb{K}}
\newcommand{\bbL}{\mathbb{L}}
\newcommand{\bbM}{\mathbb{M}}
\newcommand{\bbN}{\mathbb{N}}
\newcommand{\bbO}{\mathbb{O}}
\newcommand{\bbP}{\mathbb{P}}
\newcommand{\bbQ}{\mathbb{Q}}
\newcommand{\bbR}{\mathbb{R}}
\newcommand{\R}{\mathbb{R}}
\newcommand{\bbS}{\mathbb{S}}
\newcommand{\bbT}{\mathbb{T}}
\newcommand{\bbU}{\mathbb{U}}
\newcommand{\bbV}{\mathbb{V}}
\newcommand{\bbW}{\mathbb{W}}
\newcommand{\bbX}{\mathbb{X}}
\newcommand{\bbY}{\mathbb{Y}}
\newcommand{\bbZ}{\mathbb{Z}}
\newcommand{\bbk}{\mathbb{k}}

\newcommand{\calA}{\mathcal{A}}
\newcommand{\calB}{\mathcal{B}}
\newcommand{\calC}{\mathcal{C}}
\newcommand{\calD}{\mathcal{D}}
\newcommand{\calE}{\mathcal{E}}
\newcommand{\calF}{\mathcal{F}}
\newcommand{\calG}{\mathcal{G}}
\newcommand{\calH}{\mathcal{H}}
\newcommand{\calI}{\mathcal{I}}
\newcommand{\calJ}{\mathcal{J}}
\newcommand{\calK}{\mathcal{K}}
\newcommand{\calL}{\mathcal{L}}
\newcommand{\calM}{\mathcal{M}}
\newcommand{\calN}{\mathcal{N}}
\newcommand{\calO}{\mathcal{O}}
\newcommand{\calP}{\mathcal{P}}
\newcommand{\calQ}{\mathcal{Q}}
\newcommand{\calR}{\mathcal{R}}
\newcommand{\calS}{\mathcal{S}}
\newcommand{\calT}{\mathcal{T}}
\newcommand{\calU}{\mathcal{U}}
\newcommand{\calV}{\mathcal{V}}
\newcommand{\calW}{\mathcal{W}}
\newcommand{\calX}{\mathcal{X}}
\newcommand{\calY}{\mathcal{Y}}
\newcommand{\calZ}{\mathcal{Z}}
\newcommand{\iitem}{\vfill \item}
\newcommand{\topic}[1]{\textcolor{blue}{#1}}
\newcommand{\answerbox}{\begin{flushright}
    \begin{tikzpicture}
      \draw (0,0) rectangle (5,-1.75);
    \end{tikzpicture}\end{flushright}}
\newcommand{\solution}[1]{\textcolor{red}{#1}}
\newcommand{\points}[1]{\ [#1pts]}
%\renewcommand{\solution}[1]{}

\begin{document}
\pagestyle{fancy}


\begin{enumerate}
\item Dummit \& Foote, problem 10.3 \#2
\solution{
Suppose that $R^m \cong R^n$. Let $I$ be a maximal ideal of $R$. Then
$F^m=R^m/I^m \cong R^n/I^n =F^n$ where $F=R/I$ is a field. For finite
dimensional vector spaces, the dimension is unique, so $m=n$. On the
other hand, suppose that $m=n$. Then $R^m=R^n$, so the isomorphism is
the identity. } 
\item Let $R=\operatorname{Mat}_{n\times n}(\bbC)$, and $M=\bbC^n$
  be the $R$ module with $A\cdot \vec{v} = A\vec{v}$ (given by left
  matrix multiplication). Prove that $M$ is a cyclic $R$-module.
\solution{
Consider the vector $e_1 \in M$, the first standard basis vector. Then
$Ae_1$ is the first column of $A$. Hence, for all $v\in M$, $v=[v * * *]e_1$ where $[v *
* * ]$ is the matrix with $v$ as first column and anything else in the
other columns. Thus, $Re_1 = M$. }
\item Dummit \& Foote, problem 10.2 \#4
\solution{
Let $A$ be a $\bbZ$ module.
\begin{enumerate}
\item Suppose that $a\in A$. Then the map $\varphi_a:
  \bbZ/n\bbZ\rightarrow A$ defined by $\varphi_a([k]_n) = ka$ is
  well-defined if and only if $na=0$. Indeed, if $\varphi_a([k]_n)=ka$
  is well-defined, then $na=\varphi_a([n]_n)=\varphi_a([0]_n)=0\cdot
  a=0.$ On the other hand, suppose that $na=0$. if $k-k'=cn$ for some
  integers $k,k',c$, then $ak-ak'=acn = 0$, so $ak=ak'$, i.e.,
  $\varphi([k]_n)=\varphi([k']_n)$. 
\item Now consider $M=\operatorname{Hom}_\bbZ(\bbZ/n\bbZ, A)$. If $\varphi\in M$,
  then let $a=\varphi([1])$. Then $\varphi([k]) = ak =
  \varphi_a([k])$. Hence, $\varphi=\varphi_a$, so all elements of $M$
  are of the form $\varphi_a$ for some $a\in A$. Now define the
  function $G: M\rightarrow A_n$ vial $G(\varphi_a) =
  \varphi_a(1)=a$. By the above description, $G$ is onto. Furthermore,
  if $G(\varphi_a)=G(\varphi_b)$, then $a=b$, so $G$ is
  one-to-one. Furthemore, $G(\varphi+\phi)=(\varphi+\phi)(1) =
  \varphi(1)+\phi(1) = G(\varphi)+G(\phi)$, so $G$ is a homomorphism,
  and by the other work above, it is an isomorphism.
\end{enumerate}
}
\item Consider the ring $R=\bbC[t]/(t^2)$. Classify all
  \emph{irreducible} $R$-modules and all \emph{indecomposable}
  $R$-modules and prove your result.
\solution{
Recall that an $R$-module $M$ is a $\bbC$ vector space with the action of
a linear map $t: M\rightarrow M $ and we need to insist that $t^2=0$ since this is the case in
$R$. By Jordan Canonical Form, this means that there is a basis for
$M$ for which $t$ take the matrix form consisting of $1x1$ and $2x2$
Jordan blocks. But if $\dim M>1$, we know by Cayley's Theorem that $t$
has at least one Eigenvector, and since $t$ is nilpotent, the
Eigenvalue of this is 0. Hence, $\operatorname{span}(v)$ is a
submodule since $t.v=0v\in \operatorname{span}(v)$. Hence, if $M$ is
irreducible, $\dim M = 1$ or $\dim M=0$. If $\dim M=1$, then $t$ is a
nilpotent linear operator, so $t.v=0$ for all $v$. Thus, the only
non-zero irreducibles are $(\bbC, [0])$. 
By the above discussion, for any $R$-module $M$ with $\dim M=d$ we
can find a basis for $M$ such that the matrix of the action of $t$ is
given by
\begin{tabular}[h]{|c|c|c|c|}
\hline
$J_{n_1}(0)$ & 0 & $\dotsc$ & 0 \\ \hline 0 & $J_{n_2}(0)$ & $\dotsc$
                            & 0 \\ \hline $\vdots$ & $\vdots$ &
                                                                $\ddots$
                            & $\vdots$\\\hline 0 & 0 & $\dotsc$ &
                                                            $J_{n_k}(0)$\\
  \hline \end{tabular} where $J_{n_i}(0)$ are Jordan blocks of
Eigenvalue 0 of size at most 2. We showed in linear algebra that this
indicates $M\cong \bigoplus\limits_{i=1}^k (\bbC^{n_i}, J_{n_i}(0))$
is the direct sum decomposition of $M$. Therefore, the only
indecomposable modules are those of the form $(\bbC^2, J_2(0))$ and
$(\bbC, J_1(0))$. 
} 
\item Dummit \& Foote, problem 10.3 \#7
\solution{Suppose that $N$ and $M/N$ are finitely generated, with $N$
  generated by $\{n_1,\dotsc, n_l\}$ and $M/N$ generated by
  $\{m_1+N,\dotsc, m_k+N\}$. Then I claim that $\{n_1,\dotsc, n_l,
  m_1,\dotsc, m_k\}$ generate $M$. Indeed, let $x\in M$. Then
  \begin{align*}
    x+N &= \sum_i r_i (m_i+N)\\
&= \sum_i r_i m_i + N
  \end{align*}
since the cosets of $m_i$ generate $M/N$. Thus, $x-\sum_i r_i m_i \in
N$. But $N$ is generated by the $n_j$, so $x-\sum_i r_i m_i = \sum_j
s_j n_j$, and thus $x=\sum_i r_i m_i + \sum_j s_j n_j$. This shows
that the original set generates $M$. }
\item Recall that if $Q$ is the quiver $1\xrightarrow{a} 2$, then $FQ$
  is the $F$-vector space with basis $\{x_{e_1},x_{e_2},x_a\}$ and
  multiplication defined by concatenation of paths. We showed in class
  that a module is defined by three pieces of data: vector spaces
  $V_1$ and $V_2$, and a linear map $V_a: V_1\rightarrow
  V_2$. Classify the \emph{irreducible} $FQ$ modules. 
\solution{
Suppose that $\mathbb{V}=(V_1, V_2, V_a)$ is a module. A submodule is a
collection of subspaces $(W_1, W_2)$ with $W_i \subset V_i$ such that
$V_a(W_1) \subset W_2$. Notice that $\ker V_a\subset V_1$, and
$V_a(\ker V_a)=V_a(0)=0$, so $(\ker V_a, 0)$ is a submodule of
$\mathbb{V}$. Thus, if $V_1\neq 0$, $V_2=0$. Further, $(\bbC^n, 0, 0)$
has $(\operatorname{span}(e_1),0,0)$ as a submodule, so it is
irreducible if and only if $n=1$. \bf{Solution 1: $(\bbC^1, 0, 0)$ is
  irreducible.} Secondly, suppose that $V_2\neq 0$. Then for any
subspace $W_2 \subset V_2$, we have $(0, W_2, 0)$ a submodule (since
$V_a(0)=0\subset W_2$). So if $\dim V_2\neq 0$ for some irreducible,
then $\dim V_1=0$. Further, if $\dim V_2>1$, we can take $(0,
\operatorname{span}(e_1), 0)$ which is also a submodule, so the only
irreducible with $\dim V_2\neq 0$ is $(0, \bbC, 0)$. These are,
therefore, the only irreducible modules over $\bbC[t]$. 
}
\item Show that if $M_1$ and $M_2$ are irreducible $R$-modules and $f:
  M_1\rightarrow M_2$ is a homomorphism, then $f$ is either invertible
  of the zero map. 
\solution{We did this on another homework: $\ker f$ is a submodule of $M_1$, so
if $f$ is not zero, $\ker f=0$. Similarly, $\operatorname{im}(f)$ is a
submodule of $M_2$, so if $f$ is non-zero, $\operatorname{im}(f) =
M_2$. Thus, $f$ is an isomorphism if it is non-zero. }
\end{enumerate}
\end{document}
2018/02/14 21:22:10
