\documentclass[12pt]{article}
\usepackage{fancyhdr}
\usepackage{amsmath,amsthm,amssymb,dsfont,enumerate,color}
\usepackage[top=1in, bottom=1in]{geometry}
\usepackage{tikz}
\usepackage{tikz-3dplot}
\usetikzlibrary{patterns}
\tdplotsetmaincoords{70}{110}
\usetikzlibrary{arrows.meta}
\fancyhead[L]{MAT473}
\fancyhead[R]{Homework 1}
\fancyfoot[L]{Name: \underline{\hspace{2in}}}
\fancyfoot[R]{\large \thepage}
\fancyfoot[C]{}
\newcommand{\bbA}{\mathbb{A}}
\newcommand{\bbB}{\mathbb{B}}
\newcommand{\bbC}{\mathbb{C}}
\newcommand{\bbD}{\mathbb{D}}
\newcommand{\bbE}{\mathbb{E}}
\newcommand{\bbF}{\mathbb{F}}
\newcommand{\bbG}{\mathbb{G}}
\newcommand{\bbH}{\mathbb{H}}
\newcommand{\bbI}{\mathbb{I}}
\newcommand{\bbJ}{\mathbb{J}}
\newcommand{\bbK}{\mathbb{K}}
\newcommand{\bbL}{\mathbb{L}}
\newcommand{\bbM}{\mathbb{M}}
\newcommand{\bbN}{\mathbb{N}}
\newcommand{\bbO}{\mathbb{O}}
\newcommand{\bbP}{\mathbb{P}}
\newcommand{\bbQ}{\mathbb{Q}}
\newcommand{\bbR}{\mathbb{R}}
\newcommand{\R}{\mathbb{R}}
\newcommand{\bbS}{\mathbb{S}}
\newcommand{\bbT}{\mathbb{T}}
\newcommand{\bbU}{\mathbb{U}}
\newcommand{\bbV}{\mathbb{V}}
\newcommand{\bbW}{\mathbb{W}}
\newcommand{\bbX}{\mathbb{X}}
\newcommand{\bbY}{\mathbb{Y}}
\newcommand{\bbZ}{\mathbb{Z}}
\newcommand{\bbk}{\mathbb{k}}

\newcommand{\calA}{\mathcal{A}}
\newcommand{\calB}{\mathcal{B}}
\newcommand{\calC}{\mathcal{C}}
\newcommand{\calD}{\mathcal{D}}
\newcommand{\calE}{\mathcal{E}}
\newcommand{\calF}{\mathcal{F}}
\newcommand{\calG}{\mathcal{G}}
\newcommand{\calH}{\mathcal{H}}
\newcommand{\calI}{\mathcal{I}}
\newcommand{\calJ}{\mathcal{J}}
\newcommand{\calK}{\mathcal{K}}
\newcommand{\calL}{\mathcal{L}}
\newcommand{\calM}{\mathcal{M}}
\newcommand{\calN}{\mathcal{N}}
\newcommand{\calO}{\mathcal{O}}
\newcommand{\calP}{\mathcal{P}}
\newcommand{\calQ}{\mathcal{Q}}
\newcommand{\calR}{\mathcal{R}}
\newcommand{\calS}{\mathcal{S}}
\newcommand{\calT}{\mathcal{T}}
\newcommand{\calU}{\mathcal{U}}
\newcommand{\calV}{\mathcal{V}}
\newcommand{\calW}{\mathcal{W}}
\newcommand{\calX}{\mathcal{X}}
\newcommand{\calY}{\mathcal{Y}}
\newcommand{\calZ}{\mathcal{Z}}
\newcommand{\iitem}{\vfill \item}
\newcommand{\topic}[1]{\textcolor{blue}{#1}}
\newcommand{\answerbox}{\begin{flushright}
    \begin{tikzpicture}
      \draw (0,0) rectangle (5,-1.75);
    \end{tikzpicture}\end{flushright}}
\newcommand{\solution}[1]{\textcolor{red}{#1}}
\newcommand{\points}[1]{\ [#1pts]}
%\renewcommand{\solution}[1]{}
\newcommand{\ev}{\operatorname{ev}}

\begin{document}
\pagestyle{fancy}


\begin{enumerate}
\item Let $R$ be a ring. Prove the following basics:
  \begin{enumerate}
  \item $0\cdot a=0$ for all $a\in R$. 
\solution{Consider $0a+0a=(0+0)a=0a$. Taking the additive inverse of
  $0a$ on both sides gives $0a+0=0$ or $0a=0$. }
  \item If $a,b\in R$, then $a\cdot (-b)=-(a\cdot b)$. (Be careful as this isn't just
    ``moving around a minus sign''; it says that $a$ times the
    additive inverse of $b$ is suppose to be the additive inverse of
    $ab$, and that's what you must prove.)
\solution{$a(-b)+ab = a(-b+b)=a0=0$. Hence, $-ab=a(-b)$. }
  \end{enumerate}
\item  Let $n$ be a positive integer. Prove that the set of zero
  divisors of $\bbZ/n\bbZ$ is precisely the set of elements in
  $\bbZ/n\bbZ$ that are not relatively prime to $n$, and that the set of
  units in $\bbZ/n\bbZ$ is the set of elements that \emph{are}
  relatively prime to $n$.
\solution{
Suppose that $z\in \{1,\dotsc, n-1\}$ is an integer with
$\gcd(z,n)=d\neq 1$. Then $\operatorname{lcm}(z,n)=\frac{zn}{d}=z
\frac{n}{d}$. By definition of $d$, $\frac{n}{d}$ is an integer that
is \emph{strictly} less than $n$ since $d\neq 1$. Thus,
\begin{align*}
  [z][n/d] &= [zn/d]\\
  &= [(z/d) n]
\end{align*}
which is equal to zero since $z/d$ is an integer. Hence, if $z$ and
$n$ are not coprime, $z$ is a zero divisor. 
Now suppose that $[z]\in \bbZ/n\bbZ$ is a zero divisor. Then there is a
non-zero element $[b]\in \bbZ/n\bbZ$ such that $[bz]=[0]$ in
$\bbZ/n\bbZ$. Without loss of generality, we may assume $z,b\in
\{1,\dotsc, n-1\}$. Hence, $bz$ is a multiple of $n$, so $bz=kn$ for
some integer $k$. But since $b<n$, $zb<zn$. Therefore, the least
common multiple of $z$ and $n$ is less than $zn$. This implies that
$b$ and $n$ are \emph{not} relatively prime.  }
\solution{Suppose that $[z]\in \bbZ/n\bbZ$ is a unit. Thus, there
  is an element $[a]\in \bbZ/n\bbZ$ such that $[a][z]=[1]$, i.e.,
  $az-1$ is a multiple of $n$. Hence, $az-1=bn$ for some integer
  $b$. Equivalently, $az+bn=1$ for some integers $a$ and $b$. This
  implies that $\gcd(z,n)=1$ (since the smallest positive integer that
  can be written as a $\bbZ$-linear combination of two integers is
  their greatest common divisor). The other direction is similar.}
\item Prove that if $R$ is an integral domain and the cardinality of
  $R$ is finite, then $R$ is a field.  
\solution{
Suppose that $R$ is an integral domain of finite cardinality. Suppose
that $y\in R\setminus \{0\}$, and enumerate the elements of $R$ as
$\{x_1,\dotsc, x_n\}$. Consider the set of elements $yR =
\{yx_1,yx_2,\dotsc, yx_n\}$. Note that $\lvert yR\rvert \leq \lvert
R\rvert$ since $yR$ has at most $n$ distinct elements. Define a
function $f: R\rightarrow yR$ via $f(x)=yx$. It is clear that $\lvert yR\rvert \leq \lvert
R\rvert$ I claim first that $f$ is injective. Indeed,
suppose that $f(x_i)=f(x_j)$ for some $i,j$. Then $yx_i=yx_j$. By the
cancellation in integral domains proven in class,
$x_i=x_j$. Therefore, $f$ is injective. This implies that $\lvert R
\rvert\leq \lvert yR\rvert$. Hence, combined with the other inequality
from above, we have $\lvert yR \rvert = \lvert R\rvert$. But
$yR\subset R$ since $R$ is closed under multiplication. Since they are
of equal and finite cardinality, $yR=R$. Therefore, $1\in yR$, so
$1=yx_i$ for some $i$. }
\item Let $R$ be an integral domain, and $a,b\in R$ be elements with
  $a\neq 0$. Prove that the equation $ax^2=b$ has at most two
  solutions in $R$. Then find an example of an integral domain $R$ and
  an equation $ax^2=b$ that has no solutions, one that has exactly one
  solution, and one that has exactly two solutions.
\solution{Suppose that $a,b\in R$ with $a\neq 0$, and suppose that
  $x\neq y\in R$ are two elements with $ax^2=b$ and $ay^2=b$. Then
  $ax^2=ay^2$, so by cancellation in integral domains,
  $x^2=y^2$. Subtracting $y^2$ from both sides and factoring yields 
  \begin{align*}
    x^2-y^2 &= 0\\
(x-y)(x+y) &=0.
  \end{align*}
Since $R$ is an integral domain, this implies $x-y=0$ or
$x+y=0$. Therefore, either $x=y$ (which contradicts the assumption) or
$-x=y$. If $z$ is a third solution, then by the same reasoning $-x=z$
as well. But this implies $y=z$, so the only two possible solutions
are $x$ and $-x$. 
$\star$ Note: This does not imply that there \emph{are} two
solutions. There could be none, and there could be one. Can you
construct examples where this is the case? In $\bbZ/4\bbZ$, $x^2=1$
has exactly two solutions, $x^2=2$ has no solutions, in $\bbZ/2\bbZ$,
$x^2=1$ has only one solution.}
\item Let $\varphi: R\rightarrow R'$ be a ring homomorphism. Prove
  that $\ker \varphi$ is a subring with the additional
  \emph{absorbtion} property. That is, if $x\in \ker \varphi$ and
  $r\in R$, then $rx\in \ker \varphi$ and $xr\in \ker \varphi$. Such a
  subring is called a \emph{two-sided ideal} in $R$. Find all
  two-sided ideals in $\bbZ/60\bbZ$. 
\solution{Suppose that $x\in \ker \varphi$ and $r\in R$. Then
  $\varphi(rx)=\varphi(r)\varphi(x) = \varphi(r)\cdot 0 = 0$, and
  similarly with the other order.}
\item Consider the ring $\bbZ$ and a two-sided ideal $I\subset \bbZ$. Prove
  that $x,y\in I$ if and only if $\gcd(x,y)\in I$. 
\solution{
Suppose that $x,y\in I$. Since $I$ is an ideal, by absorption, $ax, by$ are in $I$
for any $a,b\in \bbZ$. By closure under addition, $ax+by\in I$ for all
$a,b\in \bbZ$ as well. In particular, $\gcd(x,y)$ is the smallest
positive integer that can be written in the form $ax+by$, so
$\gcd(x,y)\in I$. 
Now suppose that $d=\gcd(x,y)\in I$. Then $x=kd$ and $y=ld$ for some
$k,l\in \bbZ$. Hence, by absorption, $x,y\in I$. }
\item Let $F$ be a field and $a\in F$ be an arbitrary element. Define
  the function $\ev_a: F[x]\rightarrow F$ via $\ev_a(f) = f(a)$ (i.e.,
  just replace $x$ with $a$ and evaluate).
  \begin{enumerate}
  \item Prove that $\ev_a$ is a ring homomorphism. 
\solution{First, $\ev_a(0)=0$ since the evaluation at any point of the
  zero polynomial is zero. Now suppose that $f(x), g(x)\in F[x]$. Then
  $\ev_a(f+g)=(f+g)(a)=f(a)+g(a)=\ev_a(f)+\ev_a(g)$. Finally,
  $\ev_a(fg)=(fg)(a)=f(a)g(a)=\ev_a(f)\ev_a(g)$. }
  \item Compute the kernel of $\ev_a$ and prove your result. 
\solution{Suppose that $f\in \ker \ev_a$. Then $f(a)=0$. We can apply
  the division algorithm, dividing $f(x)$ by $(x-a)$, which yields 
  \[f(x)=(x-a)g(x)+r(x)\] where the degree of $r(x)$ is at most
  $1$. Hence, $f(x)=(x-a)g(x)+c$. From the above remarks,
  \begin{align*}
    f(a)&= (a-a)g(a)+c\\
0 &= 0+c\\
0=c.
  \end{align*}
Hence, $f(x)=(x-a)g(x)$. Therefore, $f(x)\in \ker \ev_a$ if and only
if $f(x)=(x-a)g(x)$ for some polynomial $g(x)$. }


  \end{enumerate}


\end{enumerate}
\end{document}
2018/01/03 16:59:46
