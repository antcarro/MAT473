\documentclass[12pt]{article}
\usepackage{fancyhdr}
\usepackage{amsmath,amsthm,amssymb,dsfont,enumerate,color}
\usepackage[top=1in, bottom=1in]{geometry}
\usepackage{tikz}
\usepackage{tikz-3dplot}
\usetikzlibrary{patterns}
\tdplotsetmaincoords{70}{110}
\usetikzlibrary{arrows.meta}
\fancyhead[L]{MAT473}
\fancyhead[R]{Topics}
\fancyfoot[L]{Name: \underline{\hspace{2in}}}
\fancyfoot[R]{\large \thepage}
\fancyfoot[C]{}
\newcommand{\bbA}{\mathbb{A}}
\newcommand{\bbB}{\mathbb{B}}
\newcommand{\bbC}{\mathbb{C}}
\newcommand{\bbD}{\mathbb{D}}
\newcommand{\bbE}{\mathbb{E}}
\newcommand{\bbF}{\mathbb{F}}
\newcommand{\bbG}{\mathbb{G}}
\newcommand{\bbH}{\mathbb{H}}
\newcommand{\bbI}{\mathbb{I}}
\newcommand{\bbJ}{\mathbb{J}}
\newcommand{\bbK}{\mathbb{K}}
\newcommand{\bbL}{\mathbb{L}}
\newcommand{\bbM}{\mathbb{M}}
\newcommand{\bbN}{\mathbb{N}}
\newcommand{\bbO}{\mathbb{O}}
\newcommand{\bbP}{\mathbb{P}}
\newcommand{\bbQ}{\mathbb{Q}}
\newcommand{\bbR}{\mathbb{R}}
\newcommand{\R}{\mathbb{R}}
\newcommand{\bbS}{\mathbb{S}}
\newcommand{\bbT}{\mathbb{T}}
\newcommand{\bbU}{\mathbb{U}}
\newcommand{\bbV}{\mathbb{V}}
\newcommand{\bbW}{\mathbb{W}}
\newcommand{\bbX}{\mathbb{X}}
\newcommand{\bbY}{\mathbb{Y}}
\newcommand{\bbZ}{\mathbb{Z}}
\newcommand{\bbk}{\mathbb{k}}
\newcommand{\Ab}{\mathcal{Ab}}
\newcommand{\obj}{\operatorname{obj}}
\newcommand{\mor}{\operatorname{mor}}

\newcommand{\calA}{\mathcal{A}}
\newcommand{\calB}{\mathcal{B}}
\newcommand{\calC}{\mathcal{C}}
\newcommand{\calD}{\mathcal{D}}
\newcommand{\calE}{\mathcal{E}}
\newcommand{\calF}{\mathcal{F}}
\newcommand{\calG}{\mathcal{G}}
\newcommand{\calH}{\mathcal{H}}
\newcommand{\calI}{\mathcal{I}}
\newcommand{\calJ}{\mathcal{J}}
\newcommand{\calK}{\mathcal{K}}
\newcommand{\calL}{\mathcal{L}}
\newcommand{\calM}{\mathcal{M}}
\newcommand{\calN}{\mathcal{N}}
\newcommand{\calO}{\mathcal{O}}
\newcommand{\calP}{\mathcal{P}}
\newcommand{\calQ}{\mathcal{Q}}
\newcommand{\calR}{\mathcal{R}}
\newcommand{\calS}{\mathcal{S}}
\newcommand{\calT}{\mathcal{T}}
\newcommand{\calU}{\mathcal{U}}
\newcommand{\calV}{\mathcal{V}}
\newcommand{\calW}{\mathcal{W}}
\newcommand{\calX}{\mathcal{X}}
\newcommand{\calY}{\mathcal{Y}}
\newcommand{\calZ}{\mathcal{Z}}
\newcommand{\iitem}{\vfill \item}
\newcommand{\image}{\operatorname{image}}
\newcommand{\topic}[1]{\textcolor{blue}{#1}}
\newcommand{\answerbox}{\begin{flushright}
    \begin{tikzpicture}
      \draw (0,0) rectangle (5,-1.75);
    \end{tikzpicture}\end{flushright}}
\newcommand{\solution}[1]{\textcolor{red}{#1}}
\newcommand{\points}[1]{\ [#1pts]}
%\renewcommand{\solution}[1]{}
\newcounter{excounter}[subsection]
\setcounter{excounter}{0}
\newcommand{\exercise}[1]{
\addtocounter{excounter}{1}
\textcolor{red}{\fbox{\Large \Roman{section}.\roman{subsection}.\arabic{excounter}} #1}
}
\begin{document}
\pagestyle{fancy}
\tableofcontents
\setcounter{section}{6}
\section{Basics}
\label{sec:basics}

\subsection{Introduction to rings}


\begin{description}
\item[Definition: Ring] $(R,+)$ is an abelian group, $\cdot$ is
  associative, left-and-right distributivity; can be commutative or
  not; can have identity or not (some always assume yes)
\item[Properties: Ring]\ 
  \begin{enumerate}
  \item $0a=0$ for all $a\in R$;
  \item If $1\in R$, its additive inverse is $-1$, and $-a=(-1)a$;
  \end{enumerate}
\item[Definition: division ring] A ring $R$ with unity $1\neq 0$ is a
  division ring if for each $a\in R\setminus \{0\}$, there is an
  element $b\in R$ such that $ab=ba=1$ (i.e., $R\setminus \{0\}$ is a
  group under $\cdot$). If it's commutative, $R$ is a
  field. 
\item[Examples: Rings]\ 
  \begin{enumerate}
  \item Trivial: all products are zero (can't contain identity)
  \item Zero ring
  \item Integers
  \item Rationals, reals, complex
  \item $\bbZ/n\bbZ$, but there's something to check here
  \item $\operatorname{Mat}_{n\times n}(\bbF)$
  \item Quaternion ring $\bbH = \{a+bi+cj+dk\mid a,b,c,d\in \bbR\}$
    with $ij=k$, $jk=i$, $ki=j$, and minuses if you reverse
  \item Rings of functions: $Y^X=\{f: X\rightarrow Y\}$. If $Y$ is a
    ring, then define $(f+g)(x)=f(x)+g(x)$ and
    $(fg)(x)=f(x)g(x)$. \exercise{When is it unital? When is it
      commutative?} \exercise{What if we require the functions to have
      property $F$?}
  \item Given a field $F$, $F[x_1,\dotsc, x_n]$ polynomials,
    $F(x_1,\dotsc, x_n)$ rational functions, $F[[x_1,...,x_n]]$ formal
    power series, $F[x]/(x^m)$ in which $x^m$ is set equal to zero,
    and others...
  \item Weird one... perhaps: Let $n$ be an integer, and $F$ be a
    field. Let $FQ$ denote the vector space defined on the basis
    elements $t_{ij}$ where $1\leq i \leq j \leq n$. Define
    multiplication of these elements via $t_{ij}t_{kl}=\begin{cases}
      t_{il} & \textrm{~if j=k}\\ 0 &
      \textrm{~otherwise}\end{cases}$ and extend this multiplication
    linearly. This is called the \emph{path algebra} associated to the
    directed graph $1\rightarrow 2 \rightarrow \dotsc \rightarrow n$.
  \item $R[\sqrt{d}]=\{a+b\sqrt{d} \mid a,b\in R\}$ (suppose $R$ is
    commutative) where $d$ is some integer. We make sense of
    multiplication via
    $(a+b\sqrt{d})(a'+b'\sqrt{d})=aa'+d(bb')+(ab'+ba')\sqrt{d}$ and $d
    r$ is taken to mean $\underbrace{r+r+\dotsc+r}_d$. 
  \end{enumerate}
\item[Definition: Zero divisor \& Unit] $a\in R\setminus \{0\}$ is a
  \emph{zero divisor} if $\exists b\in R$ such that $ab=0$ or
  $ba=0$. $a$ is \emph{unit} if $\exists b\in R$ such that $ab=1$. The
  set of units of $R$ is denoted $R^\times$. (Note: it is a group!)
\item[Exercise: zero divisors/units] \exercise{Prove that if $a$ is a
    unit, it is not a zero divisor, and that if $a$ is a zero divisor,
    then it is not a unit.} \exercise{Find an example of a ring $R$
    and an element $a$ where $a$ is neither a zero divisor nor a
    unit.}
\item[Exercise: units/zero divisors examples] \exercise{Describe the
    units and zero divisors of $\bbZ$.} \exercise{Describe the set of
    zero divisors of $\bbZ/n\bbZ$.} \exercise{Describe the units in
    $\bbZ/n\bbZ$.}
\item[Exercise: quadratic number field] \exercise{Show that
    $\bbQ[\sqrt{d}]$ has no zero divisors (and hence that
    $\bbZ[\sqrt{d}]$ has no zero divisors).} \exercise{Show that
    $\bbQ[\sqrt{d}]$ is a field.}
\item[Definition: integral domain] A \emph{commutative ring} with no
  zero divisors is an integral domain. 
\item[Property: cancellation in integral domains] \exercise{In an
    integral domain, $ab=0$ implies $a=0$ or $b=0$, so $ab=ac$ with
    $a\neq 0$ implies $b=c$.}
  \item[Exercise] \exercise{A finite integral domain is a field!}
  \item[Definition: subring] A subring is a subgroup that is also
    closed under multiplication. It is assumed to have the same
    identity if it has one. 
  \item[Exercises]\ 
    \begin{itemize}
    \item \exercise{The center of a ring is the set of all elements
        that commute with everything. Prove that the center is a
        subring. Prove that the center of a division ring is a field.}
    \item \exercise{Suppose that $F,G$ are two fields and $F\subset
        G$. Prove that $G$ is a vector space over $F$ (hence it makes
        sense to talk about its dimension over $F$).}
    \item \exercise{Prove that if $R$ is an integral domain, the equation $ax=b$
        has at most one solution $x$ for any pair $a,b\in R$ with $a$
        and $b$ not both equal to zero.}
    \item \exercise{Prove that if $R$ is an integral domain, there are
        at most two solutions to the equation $ax^2=b$ for any $a,b\in
        R$ with $a$ and $b$ not both equal to zero. }
      \item \exercise{Let $I$ be some indexing set, and $R_i$ be a
          ring for each $i\in I$. Define $\prod\limits_{i\in I} R_i$
          to be the direct product ring (Cartesian product with
          componentwise addition and multiplication). Prove that this
          is commutative if and only if each $R_i$ is, an integral
          domain if and only if each $R_i$ is, and a ring with unity
          if and only if each $R_i$ is.}
      \item \exercise{Consider the subset $\bigoplus\limits_{i\in I}
          R_i$ within $\prod\limits_{i\in I} R_i$ of elements
          $(r_1,r_2,\dotsc)$ in which all but finitely many of the
          $r_i$ are equal to zero. Prove that this subset is a
          ring. Explain further why this \emph{direct sum} ring does
          not have an identity if $I$ is infinite (even if each $R_i$
          does have an identity).}
      \item \exercise{Let $R$ be a commutative ring with identity, and
          $G=\{g_1,\dotsc, g_n\}$ be a finite group with operation
          written as multiplication (no matter what it really
          is). Define $RG$ to be the set of formal sums 
          \[a_1g_1+\dotsc + a_n g_n \] where $a_i\in R$ and $g_i \in
          G$. Define addition componentwise, and multiplication by
          $(ag_i)(bg_j)=(ab)(g_ig_j)$. Show that $RG$ is a ring, which
          is commutative if and only if $G$ is Abelian. }
      \item \exercise{Prove that if in the group ring $RG$, any
          element $1g_i$ is a unit, but that if $G\neq \{e\}$, there
          are always zero divisors.}
      \item \exercise{Prove that the real quaternion ring is a
          division ring.}
    \end{itemize}
\end{description}

\setcounter{subsection}{2}
\subsection{Ring homomorphisms}
\label{sec:ring-homomorphisms}

\begin{description}
\item[Definition: homomorphism/isomorphism] A ring homomorphism $\phi:
  R\rightarrow R'$ is a group homomorphism of the additive groups
  $(R,+), (R',+')$ satisfying $\phi(ab)=\phi(a)\phi(b)$ for all
  $a,b\in R$. An isomorphism is a bijective homomorphism. 
\item[Exercise: inverse homomorphism] \exercise{If $\phi: R\rightarrow
    S$ is a ring homomorphism that is bijective, then $\phi^{-1}:
    S\rightarrow R$ is also a ring homomorphism.}
\item[Definition: image and kernel] The \emph{kernel} of a ring
  homormophism is the set $\{r\in R \mid \phi(r)=0\}$. 
\item[Exercise: substructure of kernel and image] \exercise{The image
    of a ring homomorphism is a subring of the codomain.}
  \exercise{The kernel $K$ of a ring homomorphism $\phi: R\rightarrow S$
    is a subring of the domain with the additional property that $r
    k\in K$ for any $k\in K$ and $r\in R$.}
\item[Exercise: examples]\ 
  \begin{itemize}
  \item \exercise{For any positive integer $n$ define the map $\phi:
      \bbZ\rightarrow \bbZ/n\bbZ$ via $\phi(m)=[m]_n$ (the congruence
      class of $m$ modulo $n$.) Compute its kernel.}
  \item \exercise{Let $F$ be a field, and $a\in F$ be a fixed
      element. Define $\operatorname{ev}_a: F[x]\rightarrow F$ by the
      following assignment: $\operatorname{ev}_a(p(x)) = p(a)$. Prove
      that $\phi$ is a ring homomorphism. Describe the kernel.} 
  \item \exercise{Challenge: Consider the homomorphism from the
      above. Show that if $F=\bbC$, then
      $\operatorname{ker}(f)=\{p(x)\in \bbC[x] \mid p(x)=(x-a)q(x),
      \exists q(x)\in \bbC[x]\}$. }
  \item \exercise{Let $m$ be an integer. Determine the values of $m$
      for which the function $\phi_m: \bbZ\rightarrow \bbZ$ given by
      $\phi_m(n)\mapsto m\cdot n$ is a homomorphism. }
  \end{itemize}
\item[Definition: quotient rings] So the kernel $\ker \phi$ is a
  homomorphism $\phi: R\rightarrow S$ is, at core, an
  additive subgroup of the $R$. From groups, we know that it is
  normal, and the first isomorphism theorem for groups tells us that
  $R/\ker \phi$ is isomorphic (as a group) to $\image
  \phi$ via $\tilde{\phi}: r+\ker \phi\mapsto
  \phi(r)$. \exercise{Prove that $\tilde{\phi}$ is a ring
    homomorphism.}
\item[Exercise] \exercise{Let $\bbQ$ be the rational number field, and consider
  the ring $\bbQ[x]$. Define the function $\varphi: \bbQ[x]\rightarrow
  \bbR$ by $\varphi(f)=f(\sqrt{5})$.
  \begin{enumerate}
  \item Compute the kernel of $\varphi$. 
  \item Show that the image of $\varphi$ is isomorphic to
    $\bbQ[\sqrt{5}]$. 
  \end{enumerate}}
\item[Definition: ideals] Motivated by the additional structure of the
  kernel of a homomorphism, define an ideal. Let $I$ be a subgroup of
  $(R,+)$. Then $I$ is a
  \begin{itemize}
  \item \emph{left ideal} if $r x \in I$ for all pairs
    $r\in R$ and $x\in I$;
  \item \emph{right ideal} if $xr \in I$ for all pairs $r\in R$ and
    $x\in I$. 
  \item \emph{two-sided ideal} if it is both a left and right ideal. 
  \end{itemize}
we can form the quotient ring of $R$ by any two-sided ideal exactly as
we did above. 
\item[Exercise] \ Let $I\subset R$ be an ideal.
  \begin{itemize}
  \item \exercise{If there is an invertible element $x\in I$ such that
    $x^{-1}\in I$, then $I=R$. }
  \item \exercise{If $a_1,\dotsc, a_k\in I$ and $r_1,\dotsc, r_k\in
      R$ and $I$ is a left ideal, then $a_1r_1+\dotsc+a_kr_k\in I$. }
  \item \exercise{Let $I\subset \bbQ[x]$ be the set of polynomials
      whose summands are of degree at least 2. Compute the dimension
      of $\bbQ[x]/I$ as a vector space over $\bbQ$. Find a complete
      set of representatives for the elements of $\bbQ[x]/I.$}
  \item \exercise{List the ideals of $\bbZ/n\bbZ$ for each integer
      $n$. }
  \item \exercise{Prove that a commutative ring $R$ is a field if and only if it
      has no non-trivial ideals (i.e., its only ideals are $R$ and $\{0\}$.}
  \end{itemize}
\item[Exercise: examples] 
  \begin{itemize}
  \item \exercise{Describe all ideals in $\bbZ$.}
  \end{itemize}
\item[Exercise: reduction for Diophantine] The greeks (and beyond)
  were interested deeply in Diophantine equations, that is, polynomial
  equations in multiple variables with integer coefficients. The goal
  is to find all integer solutions, but this can be hard (Fermat's Last
  Theorem anyone?) Reduction modulo an integer can help because there
  are now only finitely many possibilities. \exercise{Consider the
    equation $x^2+y^2=3z^2$. Show that the only integer solution to
    this equation is the trivial one ($x=y=z=0$) by reducing modulo
    $4$ and investigating the perfect squares in this ring.}
\item[Exercise: ideal algebra] There are some nice algebraic steps
  that can be carried out with ideals. Let $I$ and $J$ be two-sided ideals:
  \begin{itemize}
  \item \exercise{Show that the sum of $I$ and $J$, defined by 
      \[I+J=\{a+b\mid a\in I,\ b\in J\}.\]}
  \item \exercise{Show that the product of $I$ and $J$, defined as the
      set of all finite sums of elements $ab$ where $a\in I$ and $b\in
      B$ is a two-sided ideal.}
\item $I^n:=I^{n-1} I$ are the powers, defined inductively. 
\item \exercise{Suppose that $\{I_j \mid j\in \Omega\}$ is a set of
    ideals in a ring $I$ (where $\Omega$ is a not-necessarily-finite
    index set). Prove that $\bigcap\limits_{j\in \Omega} I_j$ is an
    ideal.}
\item \exercise{Suppose that $I_1\subset I_2\subset \dotsc$ is a set
    of ideals in a ring $R$ that form a chain. Prove that
    $\bigcup\limits_{n=1}^\infty I_n$ is an ideal. }
  \end{itemize}
\item[Exercise: more ideal algebra] Suppose that $m,n\in \bbZ$ and
  consider the ideals $m\bbZ$ and $n\bbZ$ in $\bbZ$. \exercise{Express
    the ideal $m\bbZ+n\bbZ$ as an ideal of the form $d\bbZ$. I.e.,
    what is $d$ and why?} \exercise{Express the ideal $(m\bbZ)(n\bbZ)$
    as an ideal of the form $d\bbZ$.}
\item[Exercise: the isomorphism theorems] \ Throughout, $R$ is a ring
  with identity.
  \begin{enumerate}
  \item \exercise{Suppose that $A$ is a subring of $R$ and $I$ is an
      ideal of $R$. The set $A+I$ is defined to be the set $\{a+y\mid
      a\in A,\ y\in I\}$. Prove that 
      \[(A+I)/I \cong A/(A\cap I).\]}
  \item \exercise{Suppose that $I\subset J \subset R$ is a chain of
      ideals in $R$. Prove that $(R/I)/(J/I) \cong R/J$. }
  \item \exercise{Prove that there is a bijection between the set of
      ideals in $R$ containing $I$ and ideals in $R/I$. }
  \end{enumerate}
\end{description}

\subsection{Properties of ideals}
\label{sec:properties-ideals}

\begin{description}
\item[Definitions] Let $A$ be a subset of a ring $R$, a ring with identity.
  \begin{enumerate}
  \item We denote by $(A)$ the smallest ideal of $R$ containing
    $A$, called the \emph{ideal generated by $A$}. [Note: this is the
    intersection of all ideals that contain $A$, so by above
    exercises, it is clearly an ideal.]
  \item We have the following:
    \begin{align*}
      RA &= \{r_1a_1+\dotsc + r_n a_n \mid r_i\in R,\ a_i\in A\}\\
      AR &= \{a_1r_1+\dotsc + a_n r_n\mid r_i\in R,\ a_i\in A\}\\
      RAR&= \{r_1a_1r_1'+\dotsc + r_na_nr_n' \mid r_i, r_i'\in R,\
           a_i\in A\}.
    \end{align*}
  \item An ideal that can be generated by a single element is a
    \emph{principal ideal};
  \item An ideal that can be generated by a finite set of elements is
    called \emph{finitely generated}. 
  \end{enumerate}
\item[Exercises] \ 
  \begin{enumerate}
  \item \exercise{Prove that every ideal in $\bbZ$ is a principal
      ideal.$\star$}
  \item \exercise{Show that the ideal $(2,x)\subset \bbZ[x]$ is
      \emph{not} a principal ideal.}
  \item \exercise{Suppose that $R$ is a ring with identity and $I$ is
      a two-sided ideal in
      $R$. Prove that $I=R$ if and only if $I$ contains an invertible
      element. }
    \item \exercise{Suppose that $R$ is a commutative ring. Prove
    that $R$ is a field if and only if its only ideals are $R$ and
    $0$. As a corollary, prove that if $F$ is a field and $\varphi:
    F\rightarrow R'$ is a non-zero ring homomorphism, then $\varphi$
    is injective.}
\item \exercise{Suppose that $I$ is a two-sided ideal in the matrix ring
    $M_n(\bbC)$.} 
  \end{enumerate}
\item[Remark] If we're thinking about substructures within algebraic structures, it
often makes sense to ask about the smallest among them and the
largest. Of course, the smallest ideal in a ring $R$ is $\{0\}$, and
the largest is $R$. But perhaps we'd want to avoid those. It turns out
that the large non-trivial ones are really interesting (but you're free to think
about what the smallest might look like). 
\item[Definition: maximal] A two-sided ideal $\frak{m}\subsetneq R$ is called
  \emph{maximal} if there are no ideals $I$ with $\frak{m}\subsetneq I
  \subsetneq R$. 
\item[Exercise: Using Zorn's Lemma] \exercise{Prove that every ring
    $R$ with identity $1\neq 0$ has at least one maximal ideal. The
    proof of existence of some maximal element is often achieved using
    Zorn's lemma, which you can find in the Appendix in Dummit and
    Foote.}
\item[Exercises] \ 
  \begin{enumerate}
  \item \exercise{Suppose that $R$ is a commutative ring with
      identity. Prove that an ideal $\frak{m}\subsetneq R$ is a
      maximal ideal if and only if $R/\frak{m}$ is a field. }
This exercise is vitally important for the study of Fields and Galois
Theory. It says that if you can construct a commutative ring, $R$ and
a maximal ideal $\frak{m}$ within, then $R/\frak{m}$ is a field. As an
example, see the next exercise. 
\item \exercise{Consider the ring $\bbR[x]$. Prove that the ideal
    $(x^2+1)$ is a maximal ideal, then show that $\bbR[x]/(x^2+1)$ is
    isomorphic to a field that you're very familiar with.}
\item \exercise{Prove that $\bbZ/p\bbZ$ is a field if and only if $p$
    is a prime number.}
  \end{enumerate}
\item[Remark] Now let's think about generalizing some behavior within
  the integers. A positive integer $p>1$ is \emph{prime} if $ab=p$ implies
  $a=\pm 1$
  or $b=\pm 1$. But the really nice property is that if $ab$ is a
  multiple of $p$, then $p\mid a$ or $p mid b$.  When we cast that into the ideal corresponding to $p$, $(p)$,
  that means that if $ab\in (p)$, then $a\in (p)$ or $b\in (p)$. This
  inspires the following. 
\item[Definition: Primeness] Let $R$ be a commutative ring. An ideal
  $P\neq R$ is called \emph{prime} if $ab\in P$ implies that $a\in P$
  or $b\in P$. 
\item[Exercises] \ 
  \begin{enumerate}
  \item \exercise{Let $R=\bbQ[x,y]$. Prove that the set of polynomials
      in $R$ with 0 constant term is maximal, the ideal of all
      multiples of $x$ is prime but not maximal, and the ideal of all
      multiples of $xy$ is not prime and not maximal. }
  \item \exercise{Suppose that $R$ is a commutative ring. Prove that
      an ideal $P\subset R$ is prime if and only if $R/P$ is an
      integral domain. As a corollary, show that all maximal ideals
      are prime ideals.}
  \item \exercise{Prove that $(2,x)$ is a maximal ideal in $\bbZ[x]$.}
  \item \exercise{Let's write $F_2=\bbZ/2\bbZ$, the field with two
      elements. Consider the ideal $I$ in $F_2[x]$ generated by
      $(x^2+x+1)$. Prove that $I$ is a maximal ideal, and determine
      the addition and multiplication table for the field
      $F_2[x]/(x^2+x+1)$. Congratulations! You've witnessed a field
      with $4$ elements.}
  \end{enumerate}
\item[Concluding Remarks] What is happening here? First, recall that
  we might start by being interested in integral solutions to
  equations of the form $p(x_1,x_2,\dotsc, x_n)=0$. For example,
  $x^2+y^2-3z^2=0$. We witnessed a while back that if we were to
  reduce the coefficient set modulo 4 (i.e., reduce the polynomial via
  the ideal generated by 4) it could help us understand or restrict
  the possibilities. Think about the following progression:
  \begin{enumerate}
  \item When we only had the natural numbers, we had no solutions to
    the equations of the form $x+a=b$ when $b<a$. How frustrating! So
    let's introduce the integers. We now have additive inverses
    \emph{and} multiplication. 
  \item We notice that we have this nice property that if $a\neq 0$,
    then $ax=ay$ if and only if $x=y$. So cancellation works.
  \item But we also notice we don't \emph{always} have solutions to
    equations of the form $ax=b$, though when we do, we only have
    one. Why is that... and how could we break it? So let's introduce
    division. Now we have the rational numbers.
  \item Over the rationals, we can solve all equations of the form
    $ax=b$ when $a\neq 0$. Great!
  \item What about polynomials? A polynomial of degree $d$ with
    rational coefficients doesn't always have all of its roots
    rational, but in general there are at least $d$ of these roots
    (counting multiplicity). Cool. What do we have to do to get the
    rest of these roots? It turns out going to the reals isn't enough,
    so we actually have to pass to the complex numbers. That leap is
    two steps. The first is using analysis to pass to the real
    numbers, then using ring theory to pass to the complex numbers.
  \end{enumerate}

\end{description}

\subsection{Rings of fractions: localization}
\label{sec:rings-fractions}

Let's focus on the step from the integers to the rational numbers. The
idea is that the integers weren't robust enough to have solutions to
all equations of the form $ax=b$ where $a,b\in \bbZ$ and $a$ is
non-zero (if $a=0$, then $b$ must be zero to admit any solutions, and
we don't have a problem with that). So
we introduce inverses.
\begin{description}
\item[Remark] Fractions in the realm of integers goes like this: we
  consider the set of ordered pairs $(a,b)$ where $b\neq 0$. The first
  component is the numerator, and the second component is the
  denominator. Now we want to declare $(a,b)$ to be equivalent to
  $(c,d)$ when $ax=b$ has the same solution as $cx=d$. This would
  imply that $axd=cxb$, or $(ad-bc)x=0$. If we assume $R$ is an
  integral domain, then this means $x=0$ or $ad-bc=0$. I.e.,
  $ad=bc$. Look familiar? So we have a set of ordered pairs $\{(a,b)
  \mid a,b\in R,\ b\neq 0\}$ and an equivalence relation $(a,b)\sim
  (c,d)$ if $ad=bc$. Can we define arithmetic? 
\item[Theorem: Inversion] Suppose that $R$ is a commutative ring with identity and
  that $D\subset R$ is a \emph{multiplicative set}. I.e., if $a,b\in
  D$, then $ab\in D$, and furthermore $1\in D$. There is a ring
  $D^{-1}R$ with the following properties:
  \begin{itemize}
  \item There exists a homomorphism $\phi: R\rightarrow D^{-1}R$;
  \item If $d\in D$, then $\phi(d)$ is a unit in $D^{-1}R$;
  \item If there is any other ring $T$ and ring homomorphism $f:
    R\rightarrow T$ such that $f(d)\in T^*$ for all $d\in D$, then
    there exists a unique ring homomorphism $\tilde{f}:D^{-1}R\rightarrow T$
    such that $\tilde{f}\circ \phi = f$. 
  \end{itemize}
$D^{-1}R$ is referred to as the \emph{localization} of $R$ with
respect to $D$, or the ring of fractions of $D$ in $R$. 
\item[Exercises]\ 
  \begin{enumerate}
  \item \exercise{Consider the set $R\times D$ of ordered pairs and define a
    relation $\sim_D$ via $(r_1,d_1)\sim (r_2,d_2)$ if there exists an
    element $t\in D$ such that $t(r_1d_2-r_2d_1)=0$. Show that
    $\sim_D$ is an equivalence relation on $R\times D$.}
  \item \exercise{We define addition and multiplication on $R\times D$ in the
    way that we do on fractions:
    \begin{align*}
      (r_1,d_1)\cdot (r_2,d_2)&=(r_1r_2,d_1d_2)\\
      (r_1,d_1)+(r_2,d_2) &= (r_1d_2+r_2d_1,d_1d_2)
    \end{align*}
Prove that these operations are well-defined on $R\times D/\sim_D$. }
\item \exercise{Show that $R\times D/\sim_D$ is a ring with identity $(d,d)$ for
  any $d\in D$. (If we hadn't assumed that $R$ was a ring with
  identity, but had that $D$ was non-empty, this would give us an
  identity in $D^{-1}R$.)}
\item \exercise{Prove that the map $\phi: R\rightarrow D^{-1}R$ defined by
  $\phi(r)=(r,1)$ is a ring homomorphism and that $\phi(d)$ is a unit
  in $D^{-1}R$. }
\item \exercise{Prove that the above map is one-to-one if and only if $D$
  contains no zero-divisors. In particular, if $R$ is an integral
  domain and $0\not\in D$, this is always the case.}
\item \exercise{Lemma: Prove that if $f: R\rightarrow T$ is a ring
    homomorphism and that $f(1_R)\neq 1_T$ then
    $\operatorname{image}(f)\subset \operatorname{ZD}(T)$. In
    particular, $f(r)\not\in T^*$ for all $a\in R$.}
\item \exercise{Lemma: Prove that if $f: R\rightarrow T$ is a ring
    homomorphism with $f(1_R)=1_T$ and $d\in R^*$ then
    $f(d^{-1})=f(d)^{-1}$ (in particular, $f(d)$ is invertible).}
\item \exercise{Prove the universal property of $D^{-1}R$: If $T$ is
    any ring and $f: R\rightarrow T$ is a ring homomorphism such that
    $f(d)\in T^*$ for all $d\in D$, then there exists a unique ring
    homomorphism $\tilde{f}: D^{-1}R\rightarrow T$ such that
    $\tilde{f}\circ \phi = f$. }
  \end{enumerate}
\item[Remark] The word \emph{universal property} showed up above, and
  it might seem strange. This starts a light conversation about
  categories. Much more is to be said, but a universal property gives
  us a way to define the type of behaviour we want to see (like
  invertibility of the elements of $D$) and determine if there is a
  unique way to define this structure. 
\item[Remark] If $R$ is a field and $0\notin D$, then $D^{-1}R\cong
  R$. If $R$ is an integral domain and $D=R\setminus \{0\}$, then
  $D^{-1}R$ is a field... and this is what we wanted in the first
  place.
\item[Exercise] \exercise{Consider $\bbZ$ and the two subsets $D=2\bbZ$ and $S=\{2^n
  \mid n\in \bbZ_{\geq 0}\}$. Describe $D^{-1}\bbZ$ and
  $S^{-1}\bbZ$. Are they isomorphic? }
\item[Exercise] \exercise{If $d\in R$ is not zero and not a zero-divisor, then
  define $d^{\bbZ}=\{1,d,d^2,\dotsc,\}$. Then $(d^\bbZ)^{-1}R$ is like
  the ring of polynomials in the variable $1/d$, which we write
  $R[1/d]$. (There's something to prove here: that the powers are
  really independent.)}
\item[Exercise] \exercise{If $R$ is an integral domain and $F$ is a field
  containing $R$, then $F$ contains $\calQ R$ is isomorphic to a
  subfield of $F$. }
\item[Exercise] \exercise{Let $R$ be an integral domain and $D$ be a
  multiplicative set. Then there is a bijection between prime ideals
  in $D^{-1}R$ and prime ideals in $R$ not intersecting $D$. }
\item[Exercise] \exercise{A commutative ring is called \emph{local} if it has a
  unique maximal ideal $\mathfrak{m}$. Prove that $R-\mathfrak{m}=R^*$
  if $R$ is a local ring. }
\item[Exercise] \exercise{Consider the commutative ring with identity
    $R$ and a multiplicative set $S$. Prove that the localization
    $\varphi$ furnishes a bijection between the set of prime ideals in
    $R$ not intersecting $S$ and the prime ideals in $S^{-1}R$. }
\item[Exercise] \exercise{Let $p$ be a prime integer, and consider $S$
    the multiplicative set of integers that are \emph{not} multiples
    of $p$. (Note, this is precisely the complement of $(p)$ in
    $\bbZ$.) The localization $S^{-1}\bbZ$ is often denoted
    $\bbZ_{(p)}$. Prove that $\bbZ_{(p)}$ is a local ring with maximal
    ideal $\{\frac{t}{q} \mid t\in (p), q\in S\}$. (You may refer to
    the previous exercise.)}
\end{description}

\subsection{Interlude: Categories}
\label{sec:interl-categ-1}


A \emph{category} $\calC$ is a pair with a class of objects,
$\obj(\calC)$ and a set $\mor_\calC(A,B)$ for all pairs of objects
$A,B$ in $\calC$ such that for each triple of objects $A,B,C\in
\obj(\calC)$, there is a function $\circ: \mor_{\calC}(B,C)\times
\mor_\calC(A,B)\rightarrow \mor_\calC(A,C)$ known as
\emph{composition} that is associative on 4-tuples of objects. We
further insist that for every object $X\in \calC$, there is an element
$1_X \in \mor_\calC(X,X)$ such that $1_X\circ f = f$ and $g\circ
1_X=g$ for all $f\in \mor_\calC(Y,X)$ and $g\in \mor_\calC(X,Z)$. 

\begin{description}
\item[Examples]\
  \begin{enumerate}
  \item Let $\calC$ be the category whose objects are the positive
    integers, with $\mor_\calC (m,n)=\begin{cases} \{n/m\} & \textrm{if
        divisible}\\ \emptyset & \textrm{otherwise}\end{cases}$. So
    the \emph{morphism} set from $m$ to $n$ is either a singleton if
    $m$ doesn't divide $n$, and empty otherwise. We also need to
    furnish a \emph{composition} of functions! What should $\circ:
    \mor_\calC(m,n) \times \mor_\calC(l,m)\rightarrow \mor_\calC(l,n)$
    do? Let's assume that there \emph{is} a divisibility, that $l \mid
    m$ and $m\mid n$. Then $l\mid n$ and its morphism is $n/l$ which
    is equal to $n/m \cdot m/l$. So let's say:
    \begin{align*}
      (n/m) \circ (m/l) &= n/l
    \end{align*}
but if either set is empty, then the codomain is empty, so composition
is just the trivial map from empty set to empty set. 

This category already has an interesting object: an \emph{initial}
object, an object $\chi$ with the property that
$\mor_\calC(\chi,n)\neq \emptyset$ for all $n\in \obj(\calC)$. What is
it? Is it unique? Why? Does it have a \emph{dual} notion?
\item Let $\Ab$ be the category whose objects are finite abelian
  groups, and whose morphism sets are the set of group
  homomorphisms, with composition being... composition. Notice that the kernel of such a homomorphism is
  still a finite abelian group and the cokernel is also. This is
  called an \emph{abelian} category. 
\item Let $\mathcal{S}et$ be the category whose objects are sets and whose
  morphisms are set functions. Composition is just composition. What
  is the \emph{initial} object here? Can you realize intersection as
  an object with a universal property? What about union?
\item Vector spaces over a given field, with linear transformations. 
\item Topological spaces with continuous maps.
\item Products are universal objects, that may or may not exist:
  \fbox{Definition:} Let $I$ be an indexing set, and let $R_i$ be an
  object for each index $i\in I$. A \emph{product} of $R_i$, denoted
  $\prod\limits_{i\in I} R_i$ is an object with a morphism $p_j:
  \prod\limits_{i\in I} R_i \rightarrow R_j$ for each $j\in I$
  satisfying the following universal property: if $T$ is any other
  object in $\calC$ with morphisms $\varphi_j: T\rightarrow R_j$ for
  each $j\in J$, then there is a unique map $\Phi: \prod\limits_{i\in
    I} R_i\rightarrow T$ such that $\varphi_j \circ \Phi = p_j$ for
  each $j\in J$. 
\item You can check that the product of rings is such a universal
  object. 
\item 
  \end{enumerate}

\end{description}


\subsection{Chinese remainder theorem}
\label{sec:chin-rema-theor}

The defining structure of rings comes partly from the ring of
integers, so let's spend a little bit of time on them. We've already
seen that every ideal in $\bbZ$ is principal. In particular, every
quotient of $\bbZ$ is of the form $\bbZ/n\bbZ$. But does this
\emph{decompose} further? 
\begin{description}
\item[Definition] Ideals $A$ and $B$ in a ring $R$ are called
  \emph{comaximal} if $A+B=R$. 
\item[Theorem: Chinese Remainder Theorem] Let $A_1,\dotsc, A_k$ be
  ideals in a commutative ring $R$ with identity. Then the map
  \begin{align*}
    R& \rightarrow R/A_1 \times R/A_2 \times \dotsc \times R/A_k\\
    r &\mapsto (r+A_1,r+A_2,\dotsc, r+A_k)
  \end{align*}
is a ring homomorphism with kernel $A_1\cap A_2 \cap \dotsc \cap
A_i$. If for each pair $i,j$ with $i\neq j$ we hav $A_i$ and $A_j$ are
comaximal, then the map is surjective, and the kernel is $A_1A_2\cdots
A_k$. Thus, \[R/(A_1A_2\cdots A_k) = R/(A_1\cap A_2\cap \dotsc \cap
  A_k) \cong R/A_1 \times \dotsc \times R/A_k.\]
\item[Note] Note that if $A_i$ are principal ideals generated by
  $a_i$, then $A_1A_2\cdots A_k=(a_1a_2\cdots a_k)$. 
\item[Proof] Let's work with $k=2$. Consider $\varphi: R\rightarrow
  R/A \times R/B$ with $\varphi(r) = (r+A,r+B)$. Ring homomorphism is
  easy, the kernel is those elements with $\varphi(r)=(0+A,0+B)$ which
  means $r\in A$ and $r\in B$, so the intersection. Assuming $A$ and
  $B$ are comaximal, this means $1=a+b$ for some $a\in A$ and $b\in
  B$. But then $\varphi(r_1a+r_2b)=(r_1a+r_2(1-a) + A,
  r_1(1-b)+r_2b+B)$, which is $(r_2+A,r_1+B)$, so $\varphi$ is
  onto. Clearly $AB\subset A\cap B$ by the absorption property. Now
  suppose that $x\in A\cap B$. Then $x=x(a+b)=xa+xb$. $xa\in AB$ since
  $x\in B$, and $xb\in AB$ since $x\in A$. Thus, the result is proven.
\item[Exercise: The case of the integers] Consider the case of the integers. \exercise{Prove
    that two ideals $(n)$ and $(m)$ are comaximal if and only if
    $\gcd(m,n)=1$. } 
\item[Exercise] \exercise{Describe the ideal $(n)\cap (m)$ in
    terms of number theory.} 
\item[Exercise] \exercise{For each integer $n$, let us
    denote by $\varphi(n)$ the number units in $\bbZ/n\bbZ$. Find a
    formula for $\varphi(p^k)$ when $p$ is prime and $k$ is a positive
    integer.} 
\item[Exercise] \exercise{Prove that if $m$ and $n$ are relatively
    prime, then $\varphi(m\cdot n) = \varphi(m)\cdot \varphi(n)$. You
    may want to consider how to relate the set of units in the ring
    $R\times S$ to the set of units in $R$ and $S$ separately. }
\item[Exercise: central idempotents] Let $R$ be a ring (not necessarily commutative) with identity $1\neq 0$. An element $e\in R$ is called an
  \emph{idempotent} if $e^2=e$. \exercise{Prove that $1-e$ is also an
    idempotent, and that both $e$ and $1-e$ are non-trivial zero
    divisors when $e\neq 1$ and $e\neq 0$. }
\item[Exercise] \exercise{Prove that if
    $e$ is an idempotent such that $re=er$ for each $r\in R$, then in
    the two-sided ideal $Re$, the element $e$ is the identity.}
\item[Exercise] \exercise{Show that if $re=er$ for all $r\in R$, then $R\cong
    Re\times R(1-e)$ by using the theorem above.}
\end{description}

\section{Special Classes of Domains}
\label{sec:spec-class-doma} 

Many rings appearing in number theory have interesting structure that
we should identify and study. We only take on commutative domains in
this chapter.

\subsection{Euclidean domains}
\label{sec:euclidean-domains}

Inspired by the Euclidean algorithm, we seek integral domains that
have such an algorithm.   

\begin{description}
\item[Definition: Norm] A function $N: R\rightarrow \bbZ_{\geq 0}$
  with $N(0)=0$ is called a \emph{norm} on $R$. If, in addition,
  $N(\alpha)>0$ for all $\alpha R\setminus \{0\}$, then $N$ is a
  positive norm.
\item[Remark] Very vague, not much there. 
\item[Definition: Euclidean domain] A domain $R$ is called a
  \emph{Euclidean Domain} if it admits a norm $N$ relative to which
  $R$ has a division algorithm. That is, for any $a,b\in R$ with
  $b\neq 0$, there exist elements $q,r\in R$ such that
  \begin{align*}
    a&=qb+r\\
\operatorname{such that} r &=0\\
\operatorname{or}\ N(r) &<b
  \end{align*}
$q$ is called the \emph{quotient} and $r$ the \emph{remainder}.
\item[Examples] Clearly $\bbZ$ is an example of a Euclidean domain,
  as is $F[x]$ for any field $F$. 
\item[Euclidean Algorithm] The importance is that there is a
  Euclidean Algorithm:
  \begin{align*}
    a&=q_0b+r_0\\
    b&= q_1r_0+r_1\\
    r_0&=q_2r_1+r_2\\
    \vdots & \vdots\\
    r_{n-2} & =q_nr_{n-1}+r_n\\
r_{n-1}&= q_{n+1}r_n
  \end{align*}
where in each case we have divided $r_i$ by $r_{i+1}$ to get to the
next line. Since $N(r_1)>\dotsc > N(r_n)$, is a decreasing sequence
bounded below by zero, it must terminate, and it can only terminate if
$r_{n+1}=0$. 
\item[Why do we like it?] What's nice now is that $r_n$ can be
  expressed as an $R$ combination of $a$ and $b$ (so, for example, if
  $a$ and $b$ are in an ideal, so must $r_n$ be).
\item[Exercise] Consider the rings of the form $R=\bbZ[\sqrt{d}]$ where
  $d\in \bbZ$. Define $N:R\rightarrow \bbZ$ via $N(a+b\sqrt{d}) =
  (a+b\sqrt{d})(a-b\sqrt{d})=a^2-b^2d$. \exercise{Prove that this is a
  \emph{multiplicative norm} when $d\leq 0$. (Multiplicative means
  $N(z\cdot z')=N(z)\cdot N(z')$ for any $z,z'\in R$.)}
\item[Exercise] Consider the ring $R=\bbZ[i]$ and elements
  $\alpha=a+bi$ and $\beta=c+di$ with $\beta \neq 0$. In the field
  $\bbQ(i)$, $\frac{\alpha}{\beta} = r+si$ with $r,s\in \bbQ$ (by rationalizing the
  denominator). Round $r,s$ to the nearest integer, call it
  $\overline{r}, \overline{s}$. Then $\lvert r-\overline{r}\rvert \leq
  1/2,$ and $\lvert s-\overline{s}\rvert \leq 1/2$. Define
  $\gamma=\beta\left[(r-\overline{r})+(s-\overline{s})i\right]$. \exercise{Prove
    that $\alpha=(p+qi)\beta+\gamma$, and that $\gamma \in \bbZ[i].$}
  \exercise{Prove that $N(\gamma) \leq \frac{1}{2}N(\beta)$ (hint: use multiplicativity).}
\end{description}

\subsection{Principle ideal domain} 
\label{sec:princ-ideal-doma}

\subsection{Unique factorization domains}
\label{sec:uniq-fact-doma}

\subsection{Polynomial rings I}
\label{sec:polynomial-rings-i}

\section{Modules}
\label{sec:modules}

\subsection{Basic definitions}
\label{sec:basic-definitions}

\subsection{Module homomorphisms}
\label{sec:module-homomorphisms}

\subsection{Generation of modules, direct sums, and free modules}
\label{sec:gener-modul-direct}

\subsection{Interlude: categories}
\label{sec:interl-categ}

\begin{description}
\item[Definition: Category] 
\item[Examples] 
\item[Definition: monomorphism/epimorphism]
\item[Definition: product/coproduct] 
\end{description}

\subsection{Tensor products of modules}
\label{sec:tens-prod-modul}

\subsection{Exact sequences and special modules}
\label{sec:exact-sequ-spec}

\section{Algebraic geometry}
\label{sec:algebraic-geometry}


\end{document}
2018/01/02 02:54:54
